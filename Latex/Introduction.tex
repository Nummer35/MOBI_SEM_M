Ziel dieses Papers ist es, verschiedene Ansätze für die Anonymisierung von Spatial Trajectories vorzustellen und zu vergleichen. Hierzu folgt zunächst ein Überblick über die Thematik Location-based Services (LBS) und die damit einhergehende Problematik des Datenschutzes, anschließend daran werden ergänzende und weiterführende Arbeiten vorgestellt. Im Hauptteil des Papers werden verschiedene Methoden für die Anonymisierung von Spatial Trajectories erläutert und anhand mehrerer Aspekte miteinander verglichen. Zum Abschluss (Kurzes Stichwort für die Conclusion)

Die Kategorie der Location-based Services umfasst eine Vielzahl an mobilen Diensten, die auf Basis der Position des Nutzers selektierte Informationen und Dienste bereit stellen. Vor allem bedingt durch die zunehmende Verbreitung von Smartphones und der damit verbundenen Möglichkeit, auf Standortdaten der Nutzer zuzugreifen, ist die LBS-Branche in Deutschland in den letzten Jahren enorm gewachsen, wie eine Studie der Bayerischen Landeszentrale für neue Medien zeigt \cite{Consulting2014}. Demnach stieg alleine die Zahl der LBS-Anbieter im Zeitraum von 2012 bis 2014 von 130 auf 927 Anbieter. Dabei werden LBS vor allem in den Bereichen Tourismus (z.B. Yelp) und Verkehr (z.B. Öffi) genutzt, ebenso auch im sozialen Bereich (z.B. Foursquare, Google Latitude), im Einkauf (z.B. Kaufda, Barcoo) und im Gebiet der pervasive Games (z.B. Ingress, Geocaching, Killer).

Allen Anwendungsgebieten zu eigen ist jedoch die Problematik des Datenschutzes, welche durch die Erfordernis und Verwendung von Nutzerdaten. Dies ist besonders kritisch bei kontinuierlichen LBS, welche regelmäßige Updates der Nutzerposition erfordern, da hierbei Benutzerdaten durch räumliche und zeitliche Korrelation der Trajectories erschlossen werden können \cite{Chow2011}. Dieser Aspekt ist auch für die Nutzer relevant: Laut Studie verwenden zwar 67\% der Nutzer LBS, jedoch nur 36\% davon fühlen sich dabei sicher.

Es entsteht jedoch ein Interessenkonflikt bei den Nutzern von LBS: Auf der einen Seite steht der Wunsch nach Privatsphäre, auf der anderen Seite der Bedarf nach möglichst akkuraten und verwertbaren Ergebnissen der Dienste, welche jedoch direkt von der Qualität der übermittelten Nutzerpositionen abhängen. Ein Kompromiss zwischen diesen beiden Aspekten lässt sich durch diverse Anonymisierungsansätze erreichen, welche die Trajectories der Nutzer verschleiern, aber dennoch einen räumlichen Kontext für die Nutzung von LBS ermöglichen. Einige dieser Ansätze werden im Folgenden genauer vorgestellt.
[AB] Von Spatial Cloaking ist eine Verschleierungstaktik, bei der Anwender seinen genauen Standort mithilfe von k-anonymity [1] verheimlicht. Hierbei gibt es zwei prinzipielle Vorgehen. Zum einen kann der Anwender wenn er eine sog. Cloaked-Area erzeugen möchte bei jeder Anfrage and das LBS die selben Partner verwenden, oder er hat die Möglichkeit sich jedes mal neue Partner für die Erzeugung einer Cloaked-Area zu suchen. Jedoch ist es bei beiden Methoden nicht ausreichend, zu einem bestimmten Abfragezeitpunkt an das LBS k-anonymität zu gewährleisten [Zitat]. Im Folgenden werden zwei verschiedene Angriffe (trajectory tracing attack und anonymity-set tracing attack) auf das Spatial Cloaking Verfahren erläutert und wie der Nutzer sich gegen diese verteidigen kann. 
\subsubsection{Trajectory tracing Attacke}
Die trajectory tracing Attacke kann verwendet werden, sollte der Anwender von Spatial Cloaking immer die selben Partner zur Erzeugung einer Cloaked-Area verwenden. Dabei versucht der Angreifer das Opfer anhand seiner Anfragen an das LBS zu identifizieren indem ein Schnittpunkt zwischen dem aktuellen Gebiet und dem Gebiet in welchem er seit der letzten Anfrage sein muss errechnet wird. Angenommen es wird zum Zeitpunkt t1 eine Cloaked-Area a1 mit den Usern A, B und C erstellt und die Bewegungsgeschwindigkeit von Opfer A ist dem Angreifer bekannt. So kann der Angreifer zu jedem beliebigen nachfolgendem Zeitpunkt tn die maximale Entfernung errechnen. Daraus ergibt sich zum Zeitpunkt tn ein maximale Grenze. Zu einem Zeitpunkt t2 formt die Gruppe eine neue Spatial-Cloaking-Area a2 und stellt erneut Anfragen an das LBS. Der Angreifer hat nun die Möglichkeit die maximale Grenze vom Zeitpunkt t1 mit a2 zu schneiden und sollte nur ein Nutzer in dieser Schnittmenge enthalten sein, so hat der Angreifer User A erfolgreich identifiziert.\\ 
Der Anwender hat zwei Möglichkeiten sich gegen diese Art von Angriff zu wehren. Bei beiden Gegenmaßnahmen muss der Nutzer die maximale Grenze in Abhängigkeit des Zeitpunkt t1 berechnen. Anschließend kann er entweder die \textbf{patching technique} oder die \textbf{delaying technique} anwenden.
\begin{itemize}

\item{delaying technique} funktioniert, indem der Anwender eine Anfrage zu einem beliebigen Zeitpunkt t2 an die LBS formuliert. Würde der User nun direkt die Anfrage abschicken könnte der Angreifer die trajectory tracing Attacke durchführen und so den User identifizieren. Deswegen muss die Anfrage verzögert werden. Diese Verzögerung ist so groß, bis die maximale Grenze, bezogen auf t1, so stark angewachsen ist bis diese die gebildete Cloaked-Area a2 zum Zeitpunkt t2 komplett überlappt. Problematisch sind bei diesem Lösungsansatz jedoch, dass Anfragen an das LBS nicht zeitnah beantwortet werden können. Es gibt immer eine gewisse Wartezeit.
\item{patching technique} kombiniert die maximale Grenze basierend auf t1 mit der Cloaked-Area a2 um eine neue Cloaked-Area zu erhalten welche als Grundlage bei der Anfrage des LBS benutzt wird. Bei dieser Methode muss jedoch bei der Genauigkeit der Ergebnisse Abstriche gemacht werden. Da diese neue Cloaked-Area sehr groß werden kann. 
\end{itemize}
\subsubsection{anonymity-set tracing Attacke}
Die zweite Möglichkeit wie ein Angreifer sein Opfer identifizieren kann ist die anonymity-set tracing Attacke. Hierzu muss das Opfer bei jeder Anfrage an das LBS die Cloaked-Area mit anderen Partnern neu erzeugen. Hat der Angreifer nun zwei Anfragen an das LBS mitgelesen so kann er vergleichen welche Personen in der Cloaked-Area gleich geblieben sind und welche neu hinzu gekommen, bzw. den Bereich verlassen haben. Tritt nun der Fall ein, dass nur noch eine einzelne Person über alle Cloaked-Areas konstant bleibt, so ist diese Person das Opfer.\\
Um diesem Angriff entgegen zu wirken können drei verschiedene Verschleierungstaktiken verwendet werden. Dabei ist es nicht relevant ob es sich bei den dazu verwendeten Daten um Real-Time-Daten, also ob sich zum Zeitpunkt der Erstellung einer Cloaked-Area weitere Personen in der Nähe befinden oder nicht,  oder Historische Daten handelt. Außerdem muss ein sog. location anonymizer zwischen dem LBS und dem Anwender platziert werden. Dieser stellt die eigentliche Anfrage an das LBS um eine Anonymität des Nutzers zu gewährleisten.
\subsubsection{Group-based Ansatz}
Der group-based Ansatz ist eine Verschleierungstaktik welcher auf Real-Time Daten aufbaut. Hierbei stellt der Anwender seine Anfrage an den location anonymizer, welcher wiederum eine, mit k-1 sich in der Nähe befindenden Personen, Cloaked-Area erzeugt. Im Anschluss kann nun der location anonymizer eine Anfrage an das LBS mit der erzeugten Fläche senden. Sobald eine Antwort erhalten wird, werden die entsprechenden Informationen gefiltert und an den Abfragesteller weitergeleitet. Sollte nun die selbe Person eine zweite Abfrage an den location anonymizer senden, so wird die Cloaked-Area wieder mit den selben Personen erzeugt, welche auch schon bei der ersten Cloaked-Area involviert waren. So wird sicher gestellt, dass die anonymity-set tracing Attacke für einen potentiellen Angreifer nicht anwendbar ist.\\Wenn diese Art von Ansatz gewählt wird um seine Position zu verschlüsseln, so muss man jedoch Nachteile berücksichtigen. So muss der location anonymizer die Position von allen Nutzern zu jedem Zeitpunkt kennen. Das kann vor allem bei mobilen Endgeräten Probleme mit der Akkulaufzeit verursachen. 
\subsubsection{Distortion-based Ansatz}
Der distortion-based Ansatz ist eine Erweiterung des group-based Ansatzes. Er versucht das Problem aus dem group-based Ansatz zu beheben. Hierzu versucht der location anonymizer die Position von den Nutzern vorher zu sagen. Dazu benutzt dieser die letzte bekannt Position und Geschwindigkeit. Nun muss nur der Nutzer, welcher wirklich eine Anfrage stellt seinen Standort dem location anonymizer verraten und so kann trotzdem eine k-anonymität gewährleistet werden. 
\subsubsection{Predication-based Ansatz}
Dieser Ansatz benötigt keine Echtzeitinformationen von anderen Nutzern. Ein für ein bestimmtes Gebiet zuständiger location anonymizer speichert sich dazu die Standorte von verschiedenen Nutzern welche sich zu einer beliebigen Zeit eingeloggt haben. Soll der location anonymizer nun für einen Nutzer k-anonymität gewährleisten, hat aber nicht genug Personen in dem entsprechendem Gebiet, so greift dieser auf zuvor gespeicherte Bewegungen zurück. Ein schwerwiegendes Problem, welches sich aus diesem Ansatz ergibt, ist, dass sollte ein Angreifer in der Lage sein, zu überprüfen welche Personen sich in der Cloacked-Area befinden, sieht dieser nur das Opfer und nicht die ihm vorgespielten k Personen. 

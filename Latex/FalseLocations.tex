Der \textbf{False Locations} Ansatz basiert auf der Verwendung von Dummy Trajectories. Die Verschleierung der Position über Dummy Trajectories - also falschen, nach bestimmten Algorithmen generierten Trajectories - ist ein \textbf{Client-basierter Ansatz}, was somit einen deutlichen Unterschied zu Methoden wie \textbf{Spatial Cloaking} (siehe \ref{sec:spatialCloaking}) oder \textbf{Mix Zones} (siehe \ref{sec:mixZones}, \ref{sec:mixZonesStrasse}) darstellt.

\subsubsection{Privatsphäre-Parameter} \label{subsubsection:dgparameter}
Es werden drei Parameter vorgeschlagen, mit denen der Nutzer den Grad seiner Privatsphäre messen und bestimmten kann \cite{You2007, Lei2012}:
\begin{itemize}
	\item \textbf{Short-term or Snapshot Disclosure (SD)} basiert auf der aktuellen Position des Nutzers und der Dummies und stellt die Wahrscheinlichkeit dar, dass die reale Position des Nutzers identifiziert werden kann. Die Wahrscheinlichkeit sollte dabei unter einem spezifizierten Schwellenwert liegen. 
	Hierbei ist m die Nummer von Zeitslots in einer Trajectory, \textit{D$_{i}$} ist die das Set von realen und Dummy Trajectories zur Zeit \textit{i} und \textit{|D$_{i}$|} ist die Größe von \textit{D$_{i}$}.
	\begin{equation}
	\label{equation:SD}
	\frac{1}{m} \sum{i=1}{m}{\frac{1}{\left\lvert D_{i} \right\rvert}}
	\end{equation}	
	\item \textbf{Long-term or Trajectory Disclosure (LD)} ist die Wahrscheinlichkeit, die reale Trajectory zwischen allen vorhandenen Trajectories zu identifizieren. Basierend auf der realen Trajectory des Nutzers und den Dummy Trajectories sollte diese unter einem spezifizierten Schwellenwert liegen. 
	\textit{k} sind hierbei die Trajectories, die sich mit anderen überschneiden, und \textit{(n - k)} sind alle Trajectories, die keine Überschneidung haben. \textit{T$_{k}$} ist die Anzahl aller möglichen Trajectories, gegeben \textit{k} Trajectories.
	\begin{equation}
	\label{equation:LD}
	\frac{1}{T_{k} + \left( n - k \right)}
	\end{equation}	
	\item \textbf{Distance Deviation (DD)} beschreibt die durchschnittliche Distanz zwischen den Trajectories, wobei \textit{dist(PL$_{i}^{j}$, L$_{dk}^{j}$)} die euklidische Distanz zwischen zwei Punkten \textit{PL, P} angibt. Basierend auf der realen Trajectory des Nutzers und den Dummy Trajectories soll der durchschnittliche Distanzunterschied, also die Distanzabweichung unter den Trajectories, über einem spezifizierten Schwellenwert liegen.
	\begin{equation}
	\label{equation:DD}
	\frac{1}{m} * \frac{1}{n} * \sum{k=1}{n}{\sum{j=1}{m}{dist\left(PL_{i}^{j}, L_{dk}^{j}\right)}}
	\end{equation}	
\end{itemize}

\subsubsection{Ansätze für die Anonymisierung mit Dummies} \label{subsubsection:realdummy}
Im Folgenden werden einige Anonymisierungstechniken vorgestellt, die auf der Generierung von Dummies basieren, wobei das grundlegende Vorgehen bei allen vorgestellten Ansätzen nahezu gleich ist. Im Folgenden wird der Ablauf bei einer Anfrage mit Dummy Trajectories an ein LBS anhand eines der drei vorgestellten Ansätze geschildert \cite{Kido2005}: Ein Nutzer möchte eine Anfrage an einen LBS senden, aber möchte nicht, dass das LBS genaue Informationen über die eigene Position bekommt. Um dies zu gewährleisten setzt der Nutzer Dummy Trajectories ein. Der Schutz der Privatsphäre basiert dabei auf einem Set von Dummies, welche der Nutzer zusammen mit der realen Position übermittelt. Der Nutzer schickt eine Nachricht S an den LBS, welche die Form \textit{S = (u, L$_{1}$, L$_{2}$,..., L$_{m}$)} hat, wobei \textit{u} die ID des Nutzers ist und mit L die reale Position sowie m-1 Dummies angegeben werden. Der LBS übermittelt daraufhin eine Antwort R, bei der für jede der Positionen die dazugehörigen Informationen D$_{x}$ enthalten sind: \textit{R = ((L$_{1}$, D$_{1}$), (L$_{2}$, D$_{2}$),..., (L$_{m}$, D$_{m}$))}. Der Nutzer, der sich seiner realen Position bewusst ist, sucht sich nun aus R die erforderlichen Informationen D aus und verwirft die Resultate der zusätzlich übermittelten Dummies. Somit ist gewährleistet, dass der Nutzer der Position angepasste Informationen erhält, der LBS jedoch gleichzeitig nicht auf die reale Position des Nutzers schließen kann. Diese Vorgehensweise kann jedoch auch erweitert werden, um eine erhöhte Effizienz und Sicherheit zu erreichen:

\paragraph{Ansatz: Realitätsnahe Dummy Trajectories \cite{Kido2005}} \label{para:simple}
Dieser Ansatz konzentriert sich auf die Generierung von realitätsnahen Dummies. Wenn die Dummies willkürlich erzeugt werden - also anders als die Positionen und Bewegungen des Nutzers nicht von durch die Umgebung vorgegebene Faktoren (z.B. Bewegungsgeschwindigkeit, Wege und Straßen) eingeschränkt werden - ist die Chance hoch, dass sie auch als Dummies identifiziert werden können. Um dies zu vermeiden werden im Paper zwei verschiedene Algorithmen zur Dummy-Erzeugung vorgestellt - \textbf{Moving in a Neighborhood} und \textbf{Moving in a Limited Neighborhood} - welche in \ref{subsubsection:dgschema} näher vorgestellt werden. Um den Overhead bei dieser Technik zu reduzieren, und damit die Performanz zu verbessern, werden außerdem durch zusätzliche Maßnahmen die Kommunikationskosten gesenkt.

\paragraph{Ansatz: Front-end Modul \cite{Lu2008}} \label{para:modul}
Für den PAD-Ansatz wird zusätzlich zu den Client-seitig generierten Locations ein leichtgewichtiges Front-end Modul auf Serverseite vor der eigentlichen Suchmaschine integriert, welches die erhaltenen Positionen effizient verarbeitet. Zusätzlich dazu wird ein kompaktes Format für die übermittelten Locations entwickelt, um den Overhead niedrig zu halten. Die Dummies werden, wie in \ref{subsubsection:dgschema} dargestellt, durch die \textbf{Circle-Based Dummy Generation} und die \textbf{Grid-Based Dummy Generation} erstellt.

\paragraph{Ansatz: Pseudonyme und Middleware \cite{Sahu2012}} \label{para:middle}
Ein anderer Ansatz, um die Privatsphäre von LBS-Nutzern zu gewährleisten, wird durch die Kombination von Dummy Locations mit einem Application Server realisiert. Hierbei gibt es einige Ähnlichkeiten mit den Mix Zones \cite{Beresford2003}, die jedoch den Nachteil haben, dass durch die nicht akkurate Positionsangabe auch die Qualität der LBS abnimmt. Bei dem im Paper vorgeschlagenen Ansatz wird ebenfalls Middleware eingesetzt, welche dem Nutzer Pseudonyme zuweist, die jeweils nach einem bestimmten Zeitraum geändert werden. Anders als bei den Mix Zones übermittelt der Nutzer jedoch weiterhin Positionsupdates an die LBS, um einen bestmöglichen Service zu erhalten. Jedoch ist dadurch der alleinige Einsatz von mehreren aufeinander folgenden Pseudonymen für die Privatsphäre nicht ausreichend, da durch Inferenz eine Verknüpfung zwischen den Datensätzen hergestellt werden könnte. Deshalb kommen Dummies zum Einsatz, welche zusätzlich zu der realen Position über die Middleware an die LBS übermittelt werden. Mit diesem Vorgehen steht den LBS einerseits eine genaue Position zur Verfügung, um Ergebnisse für die Anfrage zu generieren, gleichzeitig bleibt die Identität des Nutzers jedoch unbekannt.

\subsubsection{Dummy-Generation-Schemata} \label{subsubsection:dgschema}
Im folgenden Abschnitt werden einige Algorithmen zur Generierung von Dummies vorgestellt und verglichen.

Das \textbf{Random Pattern Scheme Verfahren} \cite{You2007} beginnt mit der Auswahl eines Start- und Endpunktes. Abhängig von der Geschwindigkeit des Dummies und drei verschiedenen Bewegungstypen - horizontal, vertikal und eine Kombination beider - werden randomisiert Zellen auf einem Grid ausgewählt und daraus eine Trajectory erstellt. Die verschiedenen generierten Dummy Trajectories werden über eine längere Zeit für die Anonymisierung verwendet, um auch bei einem längeren Beobachtungszeitsaum den Schutz der Privatsphäre zu gewährleisten.

\textbf{Intersection Pattern-based Schemes} \cite{Lei2012} verfolgen das Ziel, die Anzahl an Dummies zu reduzieren und gleichzeitig die in \ref{subsubsection:dgparameter} spezifizierten Parameter einzuhalten. Dabei sollen Überschneidungen zwischen Dummies und realen Trajectories entstehen, wie bei den beiden folgenden Methoden zu sehen ist. Bei der \textbf{Rotation Dummy Generation} werden Dummies erstellt, indem die reale Nutzer Trajectory um einen Punkt rotiert wird (siehe Abbildung \ref{fig_Lei2012}). Dabei entsteht genau ein Überschneidungspunkt des Dummies mit der realen Trajectory. Da resultierenden Dummies den festgelegten Parametern SD, LD und DD entsprechen sollen, werden sie zuerst basierend auf DD generiert, indem der Rotationspunkt und -winkel variiert wird. Die resultierenden Dummies werden anschließend nach SD und LD ausgewertet, um ein Set der n besten Dummies zu erhalten, welches für die Anonymisierung verwendet wird. Damit erreichen wir, dass die benötigte Nummer an Dummies reduziert wird, während gleichzeitig die Privatsphäre-Parameter eingehalten werden. Bei dem \textbf{K-intersected Dummy Generation} Schema existieren mehrere Überschneidungspunkte \textit{k}, wobei \textit{k} vom Nutzer bestimmt werden kann. Die einzelnen Dummies werden durch einer Kombination des \textbf{Random Pattern Scheme} und der \textbf{Rotation Dummy Generation} erstellt. Die Pfade zwischen zwei Überschneidungen werden durch das \textbf{Random Pattern Scheme} generiert, die Pfade nach den Überschneidungspunkten entstehen durch \textbf{Rotation Dummy Generation}, wie in Abbildung \ref{fig_Lei2012} zu sehen.
\begin{figure}[!h]
	\centering
	\includegraphics[width=\linewidth]{Bilder/Lei2012.png}
	\caption{Rotation und K-intersected Dummy Generation, Quelle: \protect\cite{Lei2012}}
	\label{fig_Lei2012}
\end{figure}

Bei dem \textbf{Moving in a Neighborhood Algorithmus} \cite{Kido2005} werden zukünftige Dummies abhängig von der aktuellen Position des Dummies generiert (siehe Abbildung \ref{fig_Kido2005DG}). Der \textbf{Moving in a Limited Neighborhood Algorithmus} \cite{Kido2005} verwendet die selbe Methode, allerdings wird hierbei die Generierung der Dummies zusätzlich durch Daten von anderen Nutzern beeinflusst (siehe Abbildung \ref{fig_Kido2005DG}). In einer Region, die bereits eine hohe Personendichte besitzt, wird der Dummy neu generiert, da sonst die Positionen der Anfragen zu nahe beieinander liegen und damit auch der Rückschluss auf die realen Positionen einfacher ist.
\begin{figure}[!h]
	\centering
	\includegraphics[width=\linewidth]{Bilder/Kido2005DG.png}
	\caption{Moving in a (Limited) Neighborhood Generation, Quelle: \protect\cite{Kido2005}}
	\label{fig_Kido2005DG}
\end{figure}

Diese \textbf{Circle-Based Dummy Generation} \cite{Lu2008} fokussiert sich auf die Wahrung der Privatsphäre von Snapshots, also einzelner Positionen von Nutzern, und nicht von Trajectories. Eine Anwendung auf Trajectories, die ja eine Sequenz von Positionen darstellen, ist natürlich ebenfalls möglich. Die vom Algorithmus erzeugten Dummies sowie die reale Position liegen dabei alle innerhalb eines Kreises, wobei der Radius sowie der Mittelpunkt durch entsprechende Parameter randomisiert bestimmt wird (siehe Abbildung \ref{fig_Lu2008}). Die Winkel zwischen den verschieden Positionen k sind äquivalent und von der Anzahl der Positionen abhängig.

Bei der \textbf{Grid-Based Dummy Generation} \cite{Lu2008} wird ein uniformes, quadratisches Grid mit k Eckpunkten erzeugt, von denen einer der realen Position des Nutzers entspricht (siehe Abbildung \ref{fig_Lu2008}). Die k-1 erzeugten Eckpunkte dienen als Dummies für die Anfrage an LBS.
\begin{figure}[!h]
	\centering
	\subfloat[A]{\includegraphics[width=0.5\linewidth]{Bilder/Lu2008CD.png}\label{figA}}
	\subfloat[B]{\includegraphics[width=0.5\linewidth]{Bilder/Lu2008GD.png}\label{figB}}
	\caption{Circle- und Grid-based Dummy Generation, Quelle: \protect\cite{Lu2008}}
	\label{fig_Lu2008}
\end{figure}

Bei vielen Verfahren werden Umweltbedingungen (z.B. unbegehbares Gelände) und menschliches Verhalten (z.B. Pausen bei Sehenswürdigkeiten), welche Trajectories beeinflussen, gar nicht oder nur in geringem Ausmaß bei der Generierung von Dummy Trajectories beachtet, weshalb diese unter Umständen leicht identifizierbar sind. Bei \textbf{Pause-Based Dummy Generation} \cite{Kato2012} werden möglichst natürliche Dummies erzeugt, indem basierend auf bekannten Trajectories Pausenpositionen integriert werden.
Die bisher vorgestellten Ansätze arbeiten mehrheitlich mit einem \textbf{trusted anonymizer}, der als Middleware zwischen User und LBS agiert und die Anonymisierung übernimmt. Der Vorteil bei der Verwendung von Dummy Trajectories - also falschen Positionsdaten - besteht darin, dass es sich hierbei um einen \textbf{Client-basierten Ansatz} handelt. Der User ist somit für den Schutz seiner Privatsphäre selbst verantwortlich und nicht von anderen Personen abhängig (mit Ausnahme von optionaler Middleware wie in \ref{subsubsection:pseudomiddle} zu sehen). 
Der Grad des Privatsphäreschutzes ist bei dieser Methode vor allem abhängig von der Anzahl der generierten Dummies, welche in Abhängigkeit zu dem gewünschten Anonymitätsgrad und der zur Verfügung stehenden Processing Power gewählt werden muss.
Der Einsatz von Dummy Trajectories hat jedoch auch einige Nachteile. Zum einen entsteht ein Overhead bei jeder Anfrage, da sowohl beim User als auch bei den LBS viele zusätzliche Ressourcen für die Erstellung und Bearbeitung der Dummies benötigt werden. Im Gegensatz zu Dummy Positionen ist auch die Generierung von realistischen Dummy Trajectories, die nicht von realen Trajectories unterschieden werden können, relativ anspruchsvoll \cite{Beresford2003}. Abhängig von der Art des LBS können auch andere negative Effekte auftreten \cite{Beresford2005}: Durch den Overhead durch Dummies kann zum einen die Kapazität für andere, reale User eingeschränkt werden, zum anderen ist ein Einsatz der Methode bei LBS, welche pro Anfrage abrechnen, zu vermeiden. Auch bei LBS, die beispielsweise Kapazitäten überwachen (z.B. Verfügbarkeit von Parkplätzen oder Auslastung eines Raumes) ist ein Einsatz von Dummies kritisch, da hierdurch die realen Zustände verfälscht werden.

\subsubsection{Realitätsnahe Dummy Trajectories  \cite{Kido2005}} \label{subsubsection:realdummy}
Diese Anonymisierungstechnik basiert auf einem Set von Dummies, welche der User zusammen mit der realen Position übermittelt, um die eigene Privatsphäre zu schützen. Der User schickt eine Nachricht S an den LBS, welche die Form \textit{S = (u, L$_{1}$, L$_{2}$,..., L$_{m}$)} hat, wobei \textit{u} die ID des Users ist und mit L die reale Position sowie m-1 Dummies angegeben werden. Der LBS übermittelt daraufhin eine Antwort R, bei der für jede der Positionen die dazugehörigen Informationen D$_{x}$ enthalten sind: \textit{R = ((L$_{1}$, D$_{1}$), (L$_{2}$, D$_{2}$),..., (L$_{m}$, D$_{m}$))}. Der User, der sich seiner realen Position bewusst ist, sucht sich nun aus R die erforderlichen Informationen D aus. Somit ist gewährleistet, dass der User der Position angepasste Informationen erhält, der LBS jedoch gleichzeitig nicht auf die reale Position des Users schließen kann.
Eine Schwierigkeit hierbei ist die Generierung von realitätsnahen Dummies. Wenn die Dummies willkürlich erzeugt werden - also anders als die Positionen und Bewegungen des Nutzers nicht von durch die Umgebung vorgegebene Faktoren (z.B. Bewegungsgeschwindigkeit, Wege und Straßen) eingeschränkt werden - ist die Chance gut, dass sie als Dummies identifiziert werden können. Um dies zu vermeiden werden im Paper zwei verschiedene Algorithmen zur Dummy-Erzeugung vorgestellt:
\begin{itemize}
	\item Bei dem \textbf{Moving in a Neighborhood} Algorithmus werden zukünftige Dummies abhängig von der aktuellen Position des Dummies generiert.
	\item Der \textbf{Moving in a Limited Neighborhood} Algorithus verwendet die selbe Methode, allerdings wird hierbei die Generierung der Dummies zusätzlich von zusätztlichen Daten von anderen Usern beeinflusst. In einer Region mit hoher Personendichte wird der Dummy neu generiert.
\end{itemize}
Durch zusätzliche Maßnahmen werden außerdem die Kommunikationskosten gesenkt, um den Overhead bei dieser Technik zu reduzieren.

\subsubsection{Pseudonyme und Middleware \cite{Sahu2012}} \label{subsubsection:pseudomiddle}
Ein anderer Ansatz, um die Privatsphäre von LBS-Usern zu gewährleisten, wird durch die Kombination von Dummy Locations und einem Application Server realisiert. Hierbei gibt es einige Ähnlichkeiten mit den Mix Zones \cite{Beresford2003}, bei denen User über Middleware anonymisiert werden, indem sie für die Aufenthaltsdauer in einem definierten räumlichen Bereich ein Pseudonym annehmen und auf Positionsupdates verzichten. Somit kann der User nicht von anderen Personen in der Mix Zone differenziert werden, und es ebenso ist keine Verbindung zwischen dem Eintritt und Exit eines Users aus der Mix Zone herstellbar. Es ergibt sich jedoch der Nachteil, dass durch die nicht akkurate Positionsangabe auch die Qualität der LBS abnimmt. 
Bei dem im Paper vorgeschlagenen Ansatz wird ebenfalls Middleware eingesetzt, welche dem User Pseudonyme zuweist, die jeweils nach einem bestimmten Zeitraum geändert werden. Anders als bei den Mix Zones übermittelt der User jedoch weiterhin Positionsupdates an die LBS, um einen bestmöglichen Service zu erhalten. Jedoch ist dadurch der alleinige Einsatz von mehreren aufeinander folgenden Pseudonymen für die Privatsphäre nicht ausreichend, da durch Inferenz eine Verknüpfung zwischen den Datensätzen hergestellt werden könnte. Deshalb generiert der User eine Reihe von Dummy Locations, welche zusätzlich zu der realen Position über die Middleware an die LBS übermittelt werden. Damit steht den LBS eine genaue Position zur Verfügung, um Ergebnisse für die Anfrage zu generieren, gleichzeitig bleibt die Identität des Users jedoch unbekannt. Aus den Ergebnissen der Anfragen für die reale Position und den Dummy-Positionen kann der User nun die für ihn relevanten Informationen selektieren.

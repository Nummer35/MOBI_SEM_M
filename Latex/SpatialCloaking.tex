[AB] Spatial Cloaking versucht mithilfe von k-anonymity [1] zu gewährleisten, dass man vor einem LBS seine genaue Position verschleiern kann. Jedoch ist es nicht ausreichend, zu einem bestimmten Anfragezeitpunkt an das LBS k-anonymität zu gewährleisten. Im folgenden werden zwei verschiedene Angriffe (trajectory tracing attack und anonymity-set tracing attack) auf das Spatial Cloaking Verfahren erläutert und wie der Nutzer sich gegen diese verteidigen kann. 
\subsubsection{Trajectory tracing attack}
Die trajectory tracing Attacke versucht das Opfer anhand seiner Anfragen an das LBS zu identifizieren indem ein Schnittpunkt zwischen dem aktuellen Gebiet und dem Gebiet in welchem er seit der letzten Anfrage sein muss errechnet wird. Angenommen es wird zum Zeitpunkt t1 eine Spatial-Cloaking-Area a1 mit den Usern A, B und C erstellt und die Bewegungsgeschwindigkeit von Opfer A ist dem Angreifer bekannt. So kann der Angreifer zu jedem beliebigen nachfolgendem Zeitpunkt tn die maximale Entfernung errechnen. Daraus ergibt sich zum Zeitpunkt tn ein maximale Grenze. Zu einem Zeitpunkt t2 formt die Gruppe eine neue Spatial-Cloaking-Area a2 und stellt erneut Anfragen an das LBS. Der Angreifer hat nun die Möglichkeit die maximale Grenze vom Zeitpunkt t1 mit a2 zu schneiden und sollte nur ein Nutzer in dieser Schnittmenge enthalten sein, so hat der Angreifer User A erfolgreich identifiziert.
Die bisher vorgestellten Ansätze arbeiten mehrheitlich über einen \textbf{trusted anonymizer}, der als Middleware zwischen User und LBS agiert und die Anonymisierung übernimmt. Der Vorteil bei der Verwendung von Dummy Trajectories - also falschen Positionsdaten - besteht darin, dass es sich hierbei um einen \textbf{Client-basierten Ansatz} handelt. Der User ist somit für den Schutz seiner Privatsphäre selbst verantwortlich und nicht von anderen Personen oder der Verfügbarkeit des trusted anonymizers abhängig (mit Ausnahme von optionaler Middleware wie in \ref{subsubsection:pseudomiddle} zu sehen). 
Der Grad des Privatsphäreschutzes ist bei dieser Methode vor allem bedingt durch die Anzahl der generierten Dummies, welche in Abhängigkeit zu dem gewünschten Anonymitätsgrad und der zur Verfügung stehenden Processing Power gewählt werden muss.
Der Einsatz von Dummy Trajectories hat jedoch auch einige Nachteile. Zum einen entsteht ein Overhead bei jeder Anfrage, da sowohl beim User als auch bei den LBS viele zusätzliche Ressourcen für die Erstellung und Bearbeitung der Dummies benötigt werden. Abhängig von der Art des LBS können auch andere negative Effekte auftreten \cite{Beresford2005}: Durch den Overhead durch Dummies kann zum einen die Kapazität für andere, reale User eingeschränkt werden, zum anderen ist ein Einsatz der Methode bei LBS, welche pro Anfrage abrechnen, zu vermeiden. Auch bei LBS, die beispielsweise Kapazitäten überwachen (z.B. Verfügbarkeit von Parkplätzen oder Auslastung eines Raumes) ist ein Einsatz von Dummies kritisch, da hierdurch die realen Zustände verfälscht werden.
Im Gegensatz zu Dummy Positionen ist auch die Generierung von realistischen Dummy Trajectories, die nicht von realen Trajectories unterschieden werden können, relativ anspruchsvoll \cite{Beresford2003}. Allerdings ist gerade dieser Aspekt sehr wichtig, da durch die Anwendung von Data Mining Techniken auf über längere Zeit gesammelte Trajectories Rückschlüsse auf die realen Trajectories geschlossen werden können. Im Folgenden werden einige Anonymisierungstechniken vorgestellt, die auf der Generierung von Dummies basieren.



\subsubsection{Realitätsnahe Dummy Trajectories \cite{Kido2005}} \label{subsubsection:realdummy}
Diese Methode basiert auf einem Set von Dummies, welche der User zusammen mit der realen Position übermittelt, um die eigene Privatsphäre zu schützen. Der User schickt eine Nachricht S an den LBS, welche die Form \textit{S = (u, L$_{1}$, L$_{2}$,..., L$_{m}$)} hat, wobei \textit{u} die ID des Users ist und mit L die reale Position sowie m-1 Dummies angegeben werden. Der LBS übermittelt daraufhin eine Antwort R, bei der für jede der Positionen die dazugehörigen Informationen D$_{x}$ enthalten sind: \textit{R = ((L$_{1}$, D$_{1}$), (L$_{2}$, D$_{2}$),..., (L$_{m}$, D$_{m}$))}. Der User, der sich seiner realen Position bewusst ist, sucht sich nun aus R die erforderlichen Informationen D aus und verwirft die Resultate der übermittelten Dummies. Somit ist gewährleistet, dass der User der Position angepasste Informationen erhält, der LBS jedoch gleichzeitig nicht auf die reale Position des Users schließen kann.
Eine Schwierigkeit hierbei ist die Generierung von realitätsnahen Dummies. Wenn die Dummies willkürlich erzeugt werden - also anders als die Positionen und Bewegungen des Nutzers nicht von durch die Umgebung vorgegebene Faktoren (z.B. Bewegungsgeschwindigkeit, Wege und Straßen) eingeschränkt werden - ist die Chance gut, dass sie auch als Dummies identifiziert werden können. Um dies zu vermeiden werden im Paper zwei verschiedene Algorithmen zur Dummy-Erzeugung vorgestellt, \textbf{Moving in a Neighborhood} und \textbf{Moving in a Limited Neighborhood}, welche in Abschnitt \ref{subsubsection:dgschema} näher vorgestellt werden.
Um den Overhead bei dieser Technik zu reduzieren, und damit die Performanz zu verbessern, werden außerdem durch zusätzliche Maßnahmen die Kommunikationskosten gesenkt.

\subsubsection{Pseudonyme und Middleware \cite{Sahu2012}} \label{subsubsection:pseudomiddle}
Ein anderer Ansatz, um die Privatsphäre von LBS-Usern zu gewährleisten, wird durch die Kombination von Dummy Locations und einem Application Server realisiert. Hierbei gibt es einige Ähnlichkeiten mit den Mix Zones \cite{Beresford2003}, bei denen User über Middleware anonymisiert werden, indem sie für die Aufenthaltsdauer in einem definierten räumlichen Bereich ein Pseudonym annehmen und auf Positionsupdates verzichten. Somit kann der User nicht von anderen Personen in der Mix Zone differenziert werden, und es ebenso ist keine Verbindung zwischen dem Eintritt und Exit eines Users aus der Mix Zone herstellbar. Es ergibt sich jedoch der Nachteil, dass durch die nicht akkurate Positionsangabe auch die Qualität der LBS abnimmt. 
Bei dem im Paper vorgeschlagenen Ansatz wird ebenfalls Middleware eingesetzt, welche dem User Pseudonyme zuweist, die jeweils nach einem bestimmten Zeitraum geändert werden. Anders als bei den Mix Zones übermittelt der User jedoch weiterhin Positionsupdates an die LBS, um einen bestmöglichen Service zu erhalten. Jedoch ist dadurch der alleinige Einsatz von mehreren aufeinander folgenden Pseudonymen für die Privatsphäre nicht ausreichend, da durch Inferenz eine Verknüpfung zwischen den Datensätzen hergestellt werden könnte. Deshalb generiert der User eine Reihe von Dummy Locations, welche zusätzlich zu der realen Position über die Middleware an die LBS übermittelt werden. Damit steht den LBS eine genaue Position zur Verfügung, um Ergebnisse für die Anfrage zu generieren, gleichzeitig bleibt die Identität des Users jedoch unbekannt. Aus den Ergebnissen der Anfragen für die reale Position und den Dummy-Positionen kann der User nun die für ihn relevanten Informationen selektieren.

\subsubsection{Dummy-Generation-Schemata \cite{Kido2005, You2007}} \label{subsubsection:dgschema}
\begin{itemize}
	\item Bei dem \textbf{Moving in a Neighborhood} Algorithmus werden zukünftige Dummies abhängig von der aktuellen Position des Dummies generiert.
	\item Der \textbf{Moving in a Limited Neighborhood} Algorithmus verwendet die selbe Methode, allerdings wird hierbei die Generierung der Dummies zusätzlich durch Daten von anderen Usern beeinflusst. In einer Region, die bereits eine hohe Personendichte besitzt, wird der Dummy neu generiert.
\end{itemize}

\subsubsection{Privatsphäre-Parameter \cite{You2007, Lei2012}} \label{subsubsection:dgparameter}
Anhand von drei Requirements soll es Usern möglich sein, ihr eigenes Privatsphäreprofil zu definieren \cite{Lei2012}:
\begin{enumerate}
	\item Basierend auf der aktuellen Position des Users und der Dummies soll die Wahrscheinlichkeit, die reale Position des Users zu ermitteln, unter einem spezifizierten Schwellenwert liegen.
	\item Basierend auf der realen Trajectory des Users und den Dummy Trajectories soll die Wahrscheinlichkeit, die reale Trajectory des Users zu ermitteln, unter einem spezifizierten Schwellenwert liegen.
	\item Basierend auf der realen Trajectory des Users und den Dummy Trajectories soll der durchschnittliche Distanzunterschied, also die Distanzabweichung unter den Trajectories, über einem spezifizierten Schwellenwert liegen.
\end{enumerate}
Aufbauend auf diesen Requirements gibt es drei Parameter, mit denen der User den Grad seiner Privatsphäre messen und bestimmten kann \cite{You2007}:
\begin{enumerate}
	\item \textbf{Short-term or Snapshot Disclosure} ist die Wahrscheinlichkeit, dass die reale Position des Users identifiziert werden kann. Hierbei ist m die Nummer von Zeitslots in einer Trajectory, \textit{D$_{i}$} ist die das Set von realen und Dummy Trajectories zur Zeit \textit{i} und \textit{|D$_{i}$|} ist die Größe von \textit{D$_{i}$}.
	\begin{equation}
	\label{equation:SD}
	SD = \frac{1}{m} \sum{i=1}{m}{\frac{1}{\left\lvert D_{i} \right\rvert}}
	\end{equation}	
	
	\item \textbf{Long-term or Trajectory Disclosure} ist die Wahrscheinlichkeit, die reale Trajectory zwischen allen vorhandenen Trajectories zu identifizieren. \textit{k} sind hierbei die Trajectories, die sich mit anderen überschneiden, und \textit{(n - k)} sind alle Trajectories, die keine Überschneidung haben. \textit{T$_{k}$} ist die Anzahl aller möglichen Trajectories gegeben \textit{k} Trajectories.
	\begin{equation}
	\label{equation:LD}
	LD = \frac{1}{T_{k} + \left( n - k \right)}
	\end{equation}	
	
	\item \textbf{Distance Deviation} ist die durchschnittliche Distanz zwischen den Trajectories, wobei \textit{dist(PL$_{i}^{j}$, L$_{dk}^{j}$)} die euklidische Distanz zwischen zwei Punkten \textit{PL, P} angibt.
	\begin{equation}
	\label{equation:DD}
	DD = \frac{1}{m} * \frac{1}{n} * \sum{k=1}{n}{\sum{j=1}{m}{dist\left(PL_{i}^{j}, L_{dk}^{j}\right)}}
	\end{equation}	
\end{enumerate}
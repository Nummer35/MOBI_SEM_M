Dieses Paper stellt die drei Ansätze Spatial Cloaking, Mix-Zone und False Locations zum Schutz der Spatial Trajectory Privacy vor und vergleicht diese anschließend.
\begin{enumerate}
	\item In der Sektion des Spatial Cloaking werden die für diesen Ansatz spezielle Angriffsarten auf die Trajectory Privacy von LBS-Nutzern Trajectory Tracing Attacke und die Anonymity-set Attacke aufgezeigt und die daraus resultierenden Lösungsansätze Group-based, Distortion-based und Predication-based erläutert.
	\item Im Abschnitt der Mix-Zone wird das Konzept der Mix-Zone präsentiert und erläutert, unter welchen Bedingungen eine Mix-Zone die k-anonymity hält. Des weiteren werden auf Mix-Zones über Straßennetzen eingegangen, weil diese auf Grund der Einschränkung ihrer Benutzer in Bezug auf Bewegungsfreiheit und Geschwindigkeit spezielle Eigenschaften besitzen müssen.
	\item Beim Ansatz der False Locations werden zu Beginn die für diesen Ansatz geeigneten Privatsphären-Parameter Short-term und Long-term Disclosure sowie Distance Deviation vorgestellt. Anschließend werden die Ansätze Realitätsnahe Dummy Trajectories, Front-end Modul und Pseudonyme und Middleware diskutiert und verschiedene Dummy-Generation-Schemata aufgezeigt.
\end{enumerate}
Abschließend werden die verschieden Ansätze zum Schutz der Spatial Trajectory Privacy an Hand der Kriterien unterstützte LBS-Kategorien (I \& II), benötigter Rechenleistung, Nachrichteninformations-Overhead, Relocation, Involved Parties und benötigter Trusted Services verglichen. 

Dabei stellte sich heraus, dass nur bei Ansätzen mit False Locations lediglich der Benutzer selbst benötigt wurde, um dessen Anonymität zu gewährleisten. Im Gegensatz dazu brauchen die Ansätze Mix-Zone und Spatial Cloaking eine bestimmte Anzahl an Usern, um deren Privatsphäre zu schützen, weil sie auf dem Prinzip der k-anonymity beruhen. Ebenso ist auch auffällig, das die Ansätze False Location I und False Location II (siehe Tabelle \ref{table:vergleich1}) keinen Trusted Service benötigen um ihren Dienst zu erfüllen. 

Auf Grund der Benutzung von Pseudonymen können Mix-Zones und der False Location Ansatz III nur LBS der Kategorie II bedienen, im Gegensatz zu den anderen Ansätzen, die Kategorie I und II unterstützen. Auffällig ist auch, dass Mix-Zones nicht einfach relokalisiert werden können, da sie für ihre bestimmte räumliche Region angepasst sind. Auch ist bei ihnen der Nachrichteninformations-Overhead geringer als bei den anderen Ansätzen, weil der Trusted Service zwischen LBS und User nur die Anfragen und Antworten weiterleiten muss. Dem gegenüber stehen Spatial Cloaking, bei welchem eine ganze Koordinatenregion übertragen werden muss, oder False Locations, bei denen Dummy-Anfragen und aus denen resultierenden Antworten übermittelt werden.    
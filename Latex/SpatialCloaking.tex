Spatial Cloaking ist eine Verschleierungstaktik, bei der der Anwender seinen genauen Standort mithilfe von k-anonymity [1] verheimlicht. Das bedeutet, es versucht wird eine definierte Fläche zu erzeugen. Diese Fläche soll die Eigenschaft besitzen, dass mindestens k-1 andere Personen sich in dieser aufhalten können. Um dem Anwender eine einfache Möglichkeit zu bieten, seinen Standort zu verschleiern, wird ein sog. location anonymizer zwischen den Nutzer und den Locationbased Service (LBS) geschaltet. Ein location anonymizer hat die Aufgabe den Standort des Nutzers zu verheimlichen. \\ Stellt der Nutzer nun eine Anfrage an einen Service wie z. B. \glqq Zeige mir alle Restaurants in der Nähe\grqq, dann schickt das Endgerät seinen genauen Standort mit Anfrage an den location anonymizer. Dieser erzeugt nun ein Gebiet, in welchem der Nutzer sich befindet und welches gleichzeitig die Sicherheitskriterien erfüllt. Bei den Sicherheitskriterien handelt es sich primär um die größe des zu erzugenden Gebietes und wie viele weitere Personen sich in diesem aufhalten sollen. Anschließend stellt der location anonymizer die Anfrage an den vom Nutzer spezifizierten Service, jedoch wird lediglich das erzeugte Gebiet und nicht die genaue Position des Anwenders übermittelt. Als Antwort erhält der location anonymizer nun alle Restaurants in dem Gebiet und kann dem Nutzer alle daraus relevanten Informationen zur Verfügung stellen.\\ Wird im Folgenden von einem Gebiet oder Areal gesprochen, welches von einem location anonymizer erzeugt wurde, so spricht man von einer Cloaked-Area genannt. Um eine Cloaked-Area zu erzeugen gibt es verschiedene Methoden. Zum einen kann der Anonymisierer, wenn er eine sog. Cloaked-Area erzeugen möchte, bei jeder Anfrage an das LBS dieselben Partner verwenden oder zum anderen er hat die Möglichkeit sich jedes Mal neue Partner für die Erzeugung einer Cloaked-Area zu suchen. Jedoch ist es bei beiden Methoden nicht ausreichend, zu einem bestimmten Abfragezeitpunkt an den angefragten Service k-anonymität zu gewährleisten [Zitat]. Im Folgenden werden zwei verschiedene Angriffe (trajectory tracing attack und anonymity-set tracing attack) auf das Spatial Cloaking Verfahren erläutert und wie der Nutzer sich gegen diese verteidigen kann. 
\subsubsection{Trajectory tracing Attacke} 
Es ist sinnvoll die trajectory tracting Attacke zu verwenden, wenn der location anonymizer immer dieselben Partner zur Erzeugung einer Cloaked-Area verwendet. Dabei versucht der Angreifer den Anwender anhand seiner Anfragen an das LBS zu identifizieren, indem ein Schnittpunkt zwischen dem aktuellen Gebiet und dem Gebiet, in welchem er sich seit der letzten Anfrage befinden muss, errechnet wird. Angenommen es wird zum Zeitpunkt t1 eine Cloaked-Area a1 mit den Usern A, B und C erstellt und die Bewegungsgeschwindigkeit von Opfer A ist dem Angreifer bekannt. So kann der Angreifer zu jedem beliebigen nachfolgendem Zeitpunkt tn das maximale Gebiet errechnen, in welchem sich der Nutzer aufhalten kann. Dies muss unter der Berücksichtigung von Straßen- und Geländeverhältnissen geschehen. Daraus ergibt sich zum Zeitpunkt tn eine maximale Grenze. Zu einem Zeitpunkt t2 formt die Gruppe eine neue Spatial-Cloaking-Area a2 und stellt erneut Anfragen an das LBS. Der Angreifer hat nun die Möglichkeit die maximale Grenze vom Zeitpunkt t1 mit a2 zu schneiden. Sollte nur ein Nutzer in dieser Schnittmenge enthalten sein, so hat der Angreifer User A erfolgreich identifiziert.\\  Der Anonymisierer hat zwei Möglichkeiten sich gegen diese Art von Angriff zu wehren. Bei beiden Gegenmaßnahmen muss der Nutzer die maximale Grenze in Abhängigkeit des Zeitpunkt t1 berechnen. Anschließend kann er entweder die \textbf{patching technique} oder die\textbf{delaying technique} anwenden. 
\begin{itemize} 
\item{delaying technique} funktioniert, indem der Anwender eine Anfrage zu einem beliebigen Zeitpunkt t2 an die LBS formuliert. Würde der Anonymisierer nun direkt die Anfrage abschicken, dann könnte der Angreifer die trajectory tracing Attacke durchführen und so den User identifizieren. [Um dies zu verhindern?]Deswegen muss die Anfrage verzögert werden. Die Zeit der Verzögerung ist so groß[Die Verzögerung muss so groß gewählt werden, damit die maximale Grenze von t1 so stark angewachsen ist, bis...], bis die maximale Grenze, bezogen auf t1, so stark angewachsen ist bis diese die gebildete Cloaked-Area a2 zum Zeitpunkt t2 komplett überlappt. Wird nun die Schnittmenge aus a2 und der maximalen Grenze zu t1 errechnet,[so stellt diese das Ergebnis der maximalen Grenze zu t1 dar] ist das Ergebnis die maximale Grenze zu t1. Problematisch sind bei diesem Lösungsansatz jedoch, dass Anfragen an das LBS nicht zeitnah beantwortet werden können. Es gibt immer eine gewisse Wartezeit. 
\item{patching technique} kombiniert die maximale Grenze basierend auf t1 mit der Cloaked-Area a2, um eine neue Cloaked-Area zu erhalten, welche als Grundlage bei der Anfrage des LBS benutzt wird. Bei dieser Methode leidet jedoch die Genauigkeit der Ergebnisse, da diese neue Cloaked-Area sehr groß werden kann, wenn sich die Personen der Cloaked Area voneinander weg bewegen.  
\end{itemize} 
\subsubsection{anonymity-set tracing Attacke} 
In \cite{chow2007} Die zweite Möglichkeit wie ein Angreifer sein Opfer identifizieren kann, ist mithilfe der anonymity-set tracing Attacke. Hierzu muss das Opfer bei jeder Anfrage an das LBS die Cloaked-Area mit anderen Partnern neu erzeugen. Hat der Angreifer nun zwei Anfragen an das LBS mitgelesen, so kann er vergleichen, welche Personen in der Cloaked-Area gleich geblieben sind und welche neu hinzugekommen, bzw. den Bereich verlassen haben. Tritt der Fall ein, dass nur noch eine einzelne Person über alle Cloaked-Areas konstant bleibt, so ist diese Person das Opfer[der Sicherheitsattacke?].\\ Um diesem Angriff entgegen zu wirken, können drei verschiedene Verschleierungstaktiken verwendet werden. Dabei ist es nicht relevant, ob es sich bei den dazu verendeten Daten um Real-Time-Daten handelt, also ob sich zum Zeitpunkt der Erstellung einer Cloaked-Area weitere Personen in der Nähe befinden oder nicht oder[ob es sich um] historische Daten handelt. Außerdem muss ein sog. location anonymizer zwischen dem LBS und dem Anwender platziert werden. Dieser stellt die eigentliche Anfrage an das LBS, um eine Anonymität des Nutzers zu gewährleisten. 
\subsubsection{Group-based Ansatz} 
Der group-based Ansatz ist eine Verschleierungstaktik, welche auf Real-Time Daten aufbaut. Stellt dieselbe Person eine zwei [eine oder zwei?]verschiedene Abfragen an den location anonymizer, so wird die Cloaked-Area immer mit denselben Personen erzeugt, welche auch schon bei der ersten Cloaked-Area involviert waren. So wird sichergestellt, dass die anonymity-set tracing Attacke für einen potentiellen Angreifer nicht anwendbar ist.\\Wenn diese Art von Ansatz gewählt wird um seine Position zu verschlüsseln, so muss man jedoch Nachteile berücksichtigen. Der location anonymizer muss so die Position von allen Nutzern zu jedem Zeitpunkt kennen. Das kann vor allem bei mobilen Endgeräten Probleme mit der Akkulaufzeit verursachen.  
\subsubsection{Distortion-based Ansatz} 
Der distortion-based Ansatz ist eine Erweiterung des group-based Ansatzes. Er versucht das Problem des ständigen online seins aus dem group-based Ansatz zu beheben. Hierzu versucht der Anonymisierer die Position von Nutzern vorher zu sagen. Dazu benutzt dieser die letzte bekannte Position und Geschwindigkeit der einzelnen Personen. Nun muss nur der Nutzer, welcher wirklich eine Anfrage stellt, seinen Standort dem Anonymisierer verraten und es kann dadurch trotzdem eine k-anonymität gewährleistet werden. [Umgangssprache, satz besser formulieren] 
\subsubsection{Predication-based Ansatz} 
Dieser Ansatz benötigt keine Echtzeitinformationen von anderen Nutzern. Ein für ein bestimmtes Gebiet zuständiger Anonymisierer speichert sich dazu die Standorte von verschiedenen Nutzern, welche sich zu einer beliebigen Zeit eingeloggt haben. Soll der location anonymizer nun für einen Nutzer k-anonymität gewährleisten, hat aber nicht genug Personen in dem entsprechendem Gebiet, so greift dieser auf zuvor gespeicherte trajectorys zurück. Problematisch ist jedoch die optimale Fläche der Cloaked-Area zu berechnen, da ein Suchalgorithmus implementiert werden muss, welcher die besten Positionen von aufgezeichneten trajectorys ermittelt. Hier sollte mit einer guten Heuristic gearbeitet werden. 

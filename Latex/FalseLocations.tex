Die Verwendung von Dummy Trajectories hat jedoch auch einige Nachteile. Zum einen ensteht ein Overhead bei jeder Anfrage, da sowohl beim Nutzer und bei eventueller Middleware, als auch bei den LBS, viele zusätzliche Ressourcen für die Erstellung und Bearbeitung der Dummies benötigt werden.

Dummy users might have to control physical objects—opening
and closing doors, for example—or purchase services with
electronic cash. Furthermore, realistic dummy user movements
are much more difficult to construct than dummy messages.


\subsubsection{Pseudonyme und Middleware}
Ein anderer Ansatz, um die Privatsphäre von LBS-Nutzern zu gewährleisten, wird durch die Kombination von Dummy Locations und einem Application Server realisiert \cite{Sahu2012}. Hierbei gibt es einige Ähnlichkeiten mit den Mix Zones \cite{Beresford2003}, bei denen Nutzer über Middleware anonymisiert werden, indem sie für die Aufenthaltsdauer in einem definierten räumlichen Bereich ein Pseudonym annehmen und auf Positionsupdates verzichten. Somit kann der Nutzer nicht von anderen Personen in der Mix Zone differenziert werden und es ist ebenso keine Verbindung zwischen dem Eintritt und Austritt eines Nutzers aus der Mix Zone herstellbar. Es ergibt sich jedoch der Nachteil, dass durch die nicht akkurate Positionsangabe auch die Qualität der LBS abnimmt. 
Bei dem vorgestellten Ansatz wird ebenfalls Middleware eingesetzt, die dem Nutzer Pseudonyme zuweist, die nach einem bestimmten Zeitraum geändert werden. Anders als bei den Mix Zones übermittelt der Nutzer jedoch weiterhin Positionsupdates an die LBS, um einen bestmöglichen Service zu erhalten. Jedoch ist dadurch der alleinige Einsatz von mehreren aufeinander folgenden Pseudonymen für die Privatsphäre nicht ausreichend, da durch Inferenz eine Verknüpfung zwischen diesen hergestellt werden könnte. Deshalb generiert der Nutzer eine Reihe von Dummy Locations, welche zusätzlich zu der realen Position über die Middleware an die LBS übermittelt werden. Damit steht den LBS eine genaue Position zur Verfügung, um Ergebnisse für die Anfrage zu generieren, gleichzeitig bleibt die Identität des Nutzers jedoch unbekannt. Aus den Ergebnissen der Anfrage kann der Nutzer nun die für ihn relevanten Informationen selektieren.
Abhängig von der Anzahl der generierten Dummies K und den mit der Zeit wechselnden Pseudonyme N für jeden Nutzer in einer Zeit T entstehen somit K$^{N}$ verschiedene Pfade für jeden Benutzer. Die Variable K kann somit Abhängig von dem gewünschten Anonymitätsgrad und der zur Verfügung stehenden Processing Power gewählt werden.

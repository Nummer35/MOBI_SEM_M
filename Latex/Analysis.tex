Im Folgenden werden die gerade vorgestellten Ansätze anhand verschiedener Kriterien miteinander vergleichen. Die in der Tabelle \ref{table:vergleich1} aufgeführten Kriterien sind:
\begin{itemize}
	\item LBS I \cite{Chow2011}: Für die Anfrage an diese Kategorie der LBS werden konsistente Identitäten benötigt, da die Ergebnisse auf Userdaten beruhen. Ein Beispiel wäre hierfür die Suche nach Freunden, die sich gerade in der Umgebung des Users befinden.
	\item LBS II \cite{Chow2011}: Diese LBS benötigen für ihre Dienste keine Useridentitäten. Ein Geschäft, dass aktuelle Angebote an User in der Nähe sendet befindet sich beispielsweise in dieser Kategorie.
	\item Processing Power: Die im Vergleich zu einer einfachen, unverschleierten Anfrage zusätzlich benötigte Rechenpower bei Client und Servern.
	\item Overhead: Zusätzliche Informationen, die zusammen mit einer Anfrage an einen LBS übermittelt werden.
	\item Involved Parties: Personen, die für die Umsetzung der Ansätze benötigt werden.
	\item Trusted Services: Software- oder Hardwareressourcen, die zusätzlich zu Client und LBS benötigt werden.
\end{itemize}

\subsection{Spatial Cloaking}
Spatial Cloaking unterstützt aufgrund seiner Beschaffenheit LBS der Kategorie eins und zwei. Zum einen kann ein User den Service anonym anfragen, welche Kaffees in der Umgebung vorhanden sind und zum anderen ist es für Kaffeebesitzer mit dieser Verschleierungsart möglich, an alle sich in der Nähe befindenden User Coupons zu senden. Jedoch muss immer dem zwischen LBS und User geschalteten Service vollkommen vertraut werden. \\
Es wurden anhand der drei verschiedenen Ansätze (group-based Ansatz und dessen Erweiterung der distortion-based Ansatz für real-time Daten sowie der predication-based Ansatz für die Arbeit mit historischen Daten) vorgestellt, wie Spatial Cloaking implementiert werden kann. Um Angriffen auf die Anonymität der User zu verhindern, kann auf die zwei verschiedenen Techniken (delaying und patching technique) zurückgegriffen werden. Dabei muss beachtet werden, dass je nachdem, welche der Techniken zusätzlich zu dem Ansatz implementiert und verwendet werden, der Overhead und ebenfalls der Rechenaufwand steigt. Betrachtet man die Ansätze ohne zusätzliche Modifikationen, so kann davon ausgegangen werden, dass der Rechenaufwand des predication based Ansatzes höher ist als der des distortion based Ansatzes. Bei ersterem müssen die passenden footprints gesucht und anschließend möglichst gleichgroße Flächen erzeugt werden. Dieser Aufwand muss bei real time Spatial Cloaking nicht betrieben werden, da die Spatial Cloaking Gebiete durch die Nutzer fest vorgegeben sind. Der Overhead ist bei allen drei Ansätzen hoch, da teilweise extrem große Cloaked Areas zu erzeugen sind. Die Daten, die im Anschluss vom LBS an den Location Anonymizer geschickt werden, müssen dann entsprechend gefiltert und ausgewertet werden, bevor die gewünschte Antwort an den User gesendet werden kann. Spatial Cloaking bietet den Vorteil, dass zu jedem beliebigen Zeitpunkt eine Cloaked Area um einen User gebildet werden kann. Diese ist demnach theoretisch nicht auf ein bestimmtes Gebiet beschränkt, sondern wandert mit dem User mit. In der Praxis ist das Gebiet, in welchem eine Claoked Area erzeugt werden kann jedoch durch das Gebiet in dem der Location Anonymizer operiert, begrenzt. Damit ein Location Anonymizer korrekt arbeiten kann, müssen jedoch ausreichend Nutzer in der Vergangenheit vorhanden gewesen sein oder sich aktuell in dem Gebiet befinden. Spatial Cloaking eignet sich sehr gut für größere Gebiete. So kann dem Nutzer die Möglichkeit gegeben werden, seinen Standort innerhalb einer Stadt, Region oder Bundesland zu verschleiern und trotzdem akkurate Ergebnisse von LBS zu erhalten.
\subsection{Mix Zones}
Ein Unterschied gegenüber den  meisten anderen hier aufgeführten Ansätze zum Schutz der Privatsphäre bei der Verwendung von LBS ist, dass das Prinzip der Mix-Zone auf Grund der Benutzung von Pseudonymen nur LBS der zweiten Kategorie unterstützt. Ein Nachteil der dadurch entsteht ist, dass diese Methode darauf angewiesen ist, dass die LBS-Anbieter ihre Dienste so gestalten, dass diese die Verwendung von Pseudonymen unterstützen. Da sich aber viele kostenlos angebotene mobile Dienste über die Vermarktung und Auswertung von Nutzerdaten finanzieren, ist davon auszugehen, dass die meisten Dienste die Verwendung von Pseudonymen nicht unterstützen werden.

Ein Vorteil von Mix-Zones ist, dass der größte Teil der benötigten Rechenleistung vor der Nutzung des eigentlichen Services zum Berechnen und Platzieren der Mix-Zone für der zu schützenden Region aufgewendet werden kann. Somit werden für die eigentlichen Nutzung des Services der benötigte Anonymisierungsdienst weniger Rechenressourcen benötigt. \cite{Beresford2003} erwähnt aber auch die Möglichkeit, die Mix-Zones und deren Platzierung für jede Nutzergruppe bei der Verwendung des Services zu generieren. Einfluss auf die benötigte Rechenleistung zum Erstellen der Mix-Zones für eine Region sind die Größe der Region, die Form der Mix-Zone, der Grad der benötigten Anonymität und die erwartete mittlere Anzahl der Benutzer innerhalb der jeweiligen Mix-Zone. Mix-Zones für Straßennetze haben die Besonderheiten, dass die mit Mix-Zone ausreichend abzudeckende Region groß ist, dass die Form der jeweiligen Mix-Zones auf Grund der Bewegungseinschränkung der Nutzer durch das Straßennetzes komplexer sein muss, um die benötigte Anonymität zu gewährleisten, und dass die Geschwindigkeit der jeweiligen Straßenabschnitte und ermittelte Verkehrsströme in Betracht gezogen werden sollten. 

In Bezug auf den Nachrichten-Overhead ist dieser Ansatz relativ gering weil an den LBS keine zusätzlichen Information bei der Anfrage übermittelt werden, lediglich die Anfragen an den Service und die Antworten des Services müssen durch die vertrauenswürdige Middleware kopiert und weiter geleitet werden. Auch ist der Wechsel des Standort des Dienstes nicht so einfach möglich, weil im Normalfall die Platzierung und die Form der Mix-Zones für die jeweilige Region, die zu schützen ist, neu berechnet werden muss. Des weiteren werden in den Mix-Zone andere Nutzer benötigt um die Anonymität zu gewährleisten, da das Prinzip auf das Mixen der Pseudonyme der Nutzer basiert. 

\subsection{False Locations}
Die meisten Spatial Trajectory Privacy Ansätze, auch die bisher vorgestellten, arbeiten mehrheitlich über einen \textbf{trusted anonymizer}, der als Middleware zwischen User und LBS agiert und die Anonymisierung übernimmt. Der Vorteil bei der Verwendung von Dummy Trajectories - also falschen Positionsdaten - besteht darin, dass es sich hierbei um einen \textbf{Client-basierten Ansatz} handelt. Der User ist somit für den Schutz seiner Privatsphäre selbst verantwortlich und damit auch nicht abhängig von der Sicherheit und Erreichbarkeit des trusted anonymizers (mit Ausnahme von optionaler Middleware wie in \ref{para:middle} zu sehen) oder der Verfügbarkeit einer passenden Anzahl an Usern in der Umgebung für z.B. k-anonymity Ansätze. Der Grad des Privatsphäreschutzes ist bei dieser Methode vor allem bedingt durch die Anzahl der generierten Dummies, welche in Abhängigkeit zu dem gewünschten Anonymitätsgrad und der zur Verfügung stehenden Infrastruktur und Processing Power gewählt werden muss.

Der Einsatz von Dummy Trajectories hat jedoch auch einige Nachteile. Zum einen entsteht ein Overhead bei jeder Anfrage, da sowohl beim User als auch bei den LBS zusätzliche Ressourcen für Erstellung und Bearbeitung der Dummies benötigt werden. Abhängig von der Art des LBS können auch andere negative Effekte auftreten \cite{Beresford2005}: Durch den Overhead durch Dummies kann zum einen die verbleibende Kapazität für andere, reale User reduziert werden, zum anderen ist ein Einsatz der Methode bei LBS, welche pro Anfrage abrechnen, zu vermeiden. Auch bei LBS, die beispielsweise Kapazitäten überwachen (z.B. Verfügbarkeit von Parkplätzen oder Auslastung eines Raumes) ist ein Einsatz von Dummies kritisch, da hierdurch die realen Zustände verfälscht werden.

Im Gegensatz zu Dummy Positionen ist die Generierung von realistischen Dummy Trajectories, die nicht von realen Trajectories unterschieden werden können, relativ anspruchsvoll \cite{Beresford2003}. Allerdings ist gerade dieser Aspekt sehr wichtig, da über längere Zeit gesammelte Trajectories Rückschlüsse auf die realen Trajectories zulassen. 

Deshalb hat die Wahl eines geeigneten Algorithmus zum Generieren von Dummies (siehe \ref{subsubsection:dgschema}) einen erheblichen Einfluss auf die Effektivität eines Dummy Trajectory Ansatzes. Im Vergleich zur Circle- oder Grid-based Dummy Generation sind Dummies, die durch Random Pattern Schemes oder Intersection Pattern-based Schemes entstanden sind, aufgrund ihres kontinuierlichen Verlaufs besser geeignet, um die Privatsphäre des Users gegen Data Mining und längerfristige Beobachtung zu schützen. Am effizientesten erscheinen jedoch die Moving in a (Limited) Neighborhood Algorithmen und der Pause-Based Dummy Generation Algorithmus, da sie realen Trajectories von Usern sehr ähnlich sind \cite{Kukkapalli2012}. 

Vergleicht man die drei in der Tabelle \ref{table:vergleich1} dargestellten False Location Ansätze zeigt sich, dass alle für LBS der Kategorie I und II geeignet sind, abgesehen von \textit{III}. Bei dieser Methode kommen Pseudonyme zum Einsatz, weswegen sie nicht für Anfragen an LBS I geeignet ist. Bei allen drei Ansätzen wird zusätzliche Processing Power für die clientseitige Erzeugung und die serverseitige Verarbeitung der Dummy Trajectories benötigt. Hierbei wird bei \textit{II} jedoch ein Front-end Modul auf Serverseite eingesetzt um die Effizienz zu erhöhen. Bei \textit{III} wird zusätzlich noch Middleware zur Verwaltung von Pseudonymen benötigt, was die zusätzlich beanspruchte Processing Power erhöht. Der Overhead entspricht bei jedem Ansatz dem gewünschten Anonymitätsgrad k, der die Anzahl der übermittelten Dummies bestimmt. \textit{I} und \textit{II} stellen jedoch Methoden vor, die durch die der Overhead gesenkt wird. Für die Umsetzung der Ansätze wird kein anderer User außer dem Anwender benötigt, wobei bei \textit{I} zusätzliche Userdaten das Ergebnis verbessern. Trusted Services werden bei False Locations grundsätzlich nicht benötigt. Bei \textit{II} wird jedoch ein zusätzliches Modul eingesetzt, um die Effizient zu steigern, während bei \textit{III} Middleware für die Erzeugung von Synonymen eingesetzt wird.
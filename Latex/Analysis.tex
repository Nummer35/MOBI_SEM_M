Die in der Tabelle aufgeführten Ansätze werden nach verschiedenen Kriterien verglichen:
\begin{itemize}
	\item LBS I \cite{Chow2011}: Für die Anfrage an diese Kategorie der LBS werden konsistente Identitäten benötigt, da die Ergebnisse auf Userdaten beruhen. Ein Beispiel wäre hierfür die Suche nach Freunden, die sich gerade in der Umgebung des Users befinden.
	\item LBS II \cite{Chow2011}: Diese LBS benötigen für ihre Dienste keine Useridentitäten. Ein Geschäft, dass aktuelle Angebote an User in der Nähe sendet befindet sich beispielsweise in dieser Kategorie.
	\item Processing Power: Die im Vergleich zu einer einfachen, unverschleierten Anfrage zusätzlich benötigte Rechenpower bei Client und Servern.
	\item Overhead: Zusätzliche Informationen, die zusammen mit einer Anfrage an einen LBS übermittelt werden.
	\item Involved Parties: Personen, die für die Umsetzung der Ansätze benötigt werden.
	\item Trusted Services: Software- oder Hardwareressourcen, die zusätzlich zu Client und LBS benötigt werden.
\end{itemize}

\subsection{Spatial Cloaking}

\subsection{Mix Zones}
Ein Unterschied gegenüber den  meisten anderen hier aufgeführten Ansätze zum Schutz der Privatsphäre gegenüber Location-based Services ist, dass das Prinzip der Mix-Zone  auf Grund der Benutzung von Pseudonymen nur LBS der zweiten Kategorie unterstützt. Ein Nachteil der durch entsteht ist, dass diese Methode darauf angewiesen ist, dass die LBS-Anbieter ihre Dienste so gestalten, dass diese die Verwendung von Pseudonymen unterstützen. Da aber sich viele kostenlos angebotene mobile Dienste über die Vermarktung und Auswertung von Nutzerdaten Finanzieren,  ist davon aus zugehen, dass die meisten Dienste die Verwendung von Pseudonymen nicht unterstützen werden.

Ein Vorteil von Mix-Zones ist, dass der größte Teil der benötigten Rechenleistung bevor der Nutzung  des eigentlichen Services zum Berechnen und Platzieren der Mix-Zone für der zu schützenden Region aufgewendet werden kann. Und somit für die eigentlichen Nutzung des Services der benötigte Anonymisierungsdienst weniger Rechenressourcen benötigt. \cite{Beresford2003} erwähnt aber auch die Möglichkeit die Mix-Zones und deren Platzierung für jede Nutzergruppe bei der Verwendung des Services zu generieren. Einfluss auf die benötigte Rechenleistung zum Erstellen der Mix-Zones für eine Region sind die Größe der Region, die Form der Mix-Zone, der Grad der benötigten Anonymität und die erwartete mittlere Anzahl der Benutzer innerhalb der  jeweiligen Mix-Zone. Bei Mix-Zone für Straßennetze sind Besonderheiten, dass die mit Mix-Zone ausreichen abzudeckende Region groß ist, die Form der jeweiligen Mix-Zones auf Grund der Bewegungseinschränkung der Nutzer durch das Straßennetzes komplexer sein  muss, um die benötigte Anonymität zu gewährleisten, die Geschwindigkeit der jeweiligen Straßenabschnitte  und ermittelte Verkehrsströme in Betracht gezogen werden sollten. 

In Bezug auf den Nachrichten-Overhead ist dieser Ansatz relativ gering weil an den LBS keine zusätzlichen Information bei der Anfrage übermittelt werden, jediglich die Anfragen an den Service und die Antworten des Services müssen durch der zu vertrauenden Middleware kopiert und weiter geleitet werden. Auch ist  der Wechsel des Standort des Dienstes nicht so einfach möglich, weil im Normalfall die Platzierung und die Form der Mix-Zones für die jeweilige Region, die zu schützen ist, neu berechnet werden muss. Desweiteren werden in den Mix-Zone andere Nutzer benötigt um die Anonymität zu gewährleisten, da das Prinzip auf das Mixen der Pseudonyme der Nutzer basiert. 

\subsection{False Locations}
Die meisten Spatial Trajectory Privacy Ansätze, auch die bisher vorgestellten, arbeiten mehrheitlich über einen \textbf{trusted anonymizer}, der als Middleware zwischen User und LBS agiert und die Anonymisierung übernimmt. Der Vorteil bei der Verwendung von Dummy Trajectories - also falschen Positionsdaten - besteht darin, dass es sich hierbei um einen \textbf{Client-basierten Ansatz} handelt. Der User ist somit für den Schutz seiner Privatsphäre selbst verantwortlich und damit auch nicht abhängig von der Sicherheit und Erreichbarkeit des trusted anonymizers (mit Ausnahme von optionaler Middleware wie in \ref{para:middle} zu sehen) oder der Verfügbarkeit einer passenden Anzahl an Usern in der Umgebung für z.B. k-anonymity Ansätze. Der Grad des Privatsphäreschutzes ist bei dieser Methode vor allem bedingt durch die Anzahl der generierten Dummies, welche in Abhängigkeit zu dem gewünschten Anonymitätsgrad und der zur Verfügung stehenden Infrastruktur und Processing Power gewählt werden muss.

Der Einsatz von Dummy Trajectories hat jedoch auch einige Nachteile. Zum einen entsteht ein Overhead bei jeder Anfrage, da sowohl beim User als auch bei den LBS zusätzliche Ressourcen für Erstellung und Bearbeitung der Dummies benötigt werden. Abhängig von der Art des LBS können auch andere negative Effekte auftreten \cite{Beresford2005}: Durch den Overhead durch Dummies kann zum einen die verbleibende Kapazität für andere, reale User reduziert werden, zum anderen ist ein Einsatz der Methode bei LBS, welche pro Anfrage abrechnen, zu vermeiden. Auch bei LBS, die beispielsweise Kapazitäten überwachen (z.B. Verfügbarkeit von Parkplätzen oder Auslastung eines Raumes) ist ein Einsatz von Dummies kritisch, da hierdurch die realen Zustände verfälscht werden.

Im Gegensatz zu Dummy Positionen ist die Generierung von realistischen Dummy Trajectories, die nicht von realen Trajectories unterschieden werden können, relativ anspruchsvoll \cite{Beresford2003}. Allerdings ist gerade dieser Aspekt sehr wichtig, da durch die Anwendung von Data Mining Techniken auf über längere Zeit gesammelte Trajectories Rückschlüsse auf die realen Trajectories geschlossen werden können. 

Die Wahl eines geeigneten Algorithmus zum Generieren von Dummies (siehe: \ref{subsubsection:dgschema}) hat einen erheblichen Einfluss auf die Effektivität eines Dummy Trajectory Ansatzes. Im Vergleich zur Circle- oder Grid-based Dummy Generation sind Dummies, die durch Random Pattern Schemes oder Intersection Pattern-based Schemes entstanden sind, aufgrund ihres kontinuierlichen Verlaufs besser geeignet, um die Privatsphäre des Users gegen Data Mining und längerfristige Beobachtung zu schützen. Am effizientesten sind jedoch die Moving in a (Limited) Neighborhood Algorithmen und der Pause-Based Dummy Generation Algorithmus, da sie realen Trajectories von Usern sehr ähnlich sind \cite{Kukkapalli2012}. 

Vergleicht man die drei in der Tabelle \ref{table:vergleich1} dargestellten False Location Ansätze zeigt sich, dass alle für LBS der Kategorie I und II geeignet sind, abgesehen von III. Bei dieser Methode kommen Pseudonyme zum Einsatz, weswegen sie nicht für Anfragen an LBS I geeignet ist. Bei allen drei Ansätzen wird zusätzliche Processing Power für die clientseitige Erzeugung und die serverseitige Verarbeitung der Dummy Trajectories benötigt. Hierbei wird bei II jedoch ein Front-end Modul auf Serverseite eingesetzt um die Effizienz zu erhöhen. Bei III wird zusätzlich noch Middleware zur Verwaltung von Pseudonymen benötigt, was die zusätzlich beanspruchte Processing Power erhöht. Der Overhead entspricht bei jedem Ansatz dem gewünschten Anonymitätsgrad k, der die Anzahl der übermittelten Dummies bestimmt. I und II stellen jedoch Methoden vor, die durch die der Overhead gesenkt wird. Für die Umsetzung der Ansätze wird kein anderer User außer dem Anwender benötigt, wobei bei I zusätzliche Userdaten das Ergebnis verbessern. Trusted Services werden bei False Locations grundsätzlich nicht benötigt. Bei II wird jedoch ein zusätzliches Modul eingesetzt, um die Effizient zu steigern, während bei III Middleware für die Erzeugung von Synonymen eingesetzt wird.
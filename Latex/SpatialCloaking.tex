[AB] Spatial Cloaking versucht mithilfe von k-anonymity [1] zu gewährleisten, dass man vor einem LBS seine genaue Position verschleiern kann. Hierbei gibt es zwei prinzipielle Vorgehen. Zum einen kann der Anwender wenn er eine Cloaked-Area erzeugen möchte bei jeder Anfrage and das LBS die selben Partner verwenden. Zum anderen hat er die Möglichkeit sich jedes mal neue Partner für eine Cloaked-Area zu suchen. Jedoch ist es bei beiden Methoden nicht ausreichend, zu einem bestimmten Anfragezeitpunkt an das LBS k-anonymität zu gewährleisten [Zitat]. Im folgenden werden zwei verschiedene Angriffe (trajectory tracing attack und anonymity-set tracing attack) auf das Spatial Cloaking Verfahren erläutert und wie der Nutzer sich gegen diese verteidigen kann. 
\subsubsection{Trajectory tracing attack}
Die trajectory tracing Attacke kann verwendet werden, sollte der Anwender von Spatial Cloaking immer die selben Partner zur Erzeugung einer Cloaked-Area verwenden. Dabei versucht der Angreifer das Opfer anhand seiner Anfragen an das LBS zu identifizieren indem ein Schnittpunkt zwischen dem aktuellen Gebiet und dem Gebiet in welchem er seit der letzten Anfrage sein muss errechnet wird. Angenommen es wird zum Zeitpunkt t1 eine Cloaked-Area a1 mit den Usern A, B und C erstellt und die Bewegungsgeschwindigkeit von Opfer A ist dem Angreifer bekannt. So kann der Angreifer zu jedem beliebigen nachfolgendem Zeitpunkt tn die maximale Entfernung errechnen. Daraus ergibt sich zum Zeitpunkt tn ein maximale Grenze. Zu einem Zeitpunkt t2 formt die Gruppe eine neue Spatial-Cloaking-Area a2 und stellt erneut Anfragen an das LBS. Der Angreifer hat nun die Möglichkeit die maximale Grenze vom Zeitpunkt t1 mit a2 zu schneiden und sollte nur ein Nutzer in dieser Schnittmenge enthalten sein, so hat der Angreifer User A erfolgreich identifiziert.\\ 
Der Anwender hat zwei Möglichkeiten sich gegen diese Art von Angriff zu wehren. Bei beiden Gegenmaßnahmen muss der Nutzer die maximale Grenze in Abhänigkeit des Zeitpunkt t1 berechnen. Anschließend kann er entweder die \textbf{patching technique} oder die \textbf{delaying technique} anwenden.
\begin{itemize}

\item{delaying technique} funktioniert, indem der Anwender eine Anfrage zu einem beliebigen Zeitpunkt t2 an die LBS formuliert. Würde der User nun direkt die Anfrage abschicken könnte der Angreifer die trajectory tracing attack durchführen und so den User identifizieren. Deswegen muss die Anfrage verzögert werden. Diese Verzögerung ist so groß, bis die maximale Grenze, bezogen auf t1, so stark angewachsen ist bis diese die gebildete Cloaked-Area a2 zum Zeitpunkt t2 komplett überlappt. Problematisch sind bei diesem Lösungsansatz jedoch, dass Anfragen an das LBS nicht zeitnah beantwortet werden können. Es gibt immer eine gewisse Wartezeit.
\item{patching technique} kombiniert die maximale Grenze basierend auf t1 mit der Cloaked-Area a2 um eine neue Cloaked-Area zu erhalten welche als Grundlage bei der Anfrage des LBS benutzt wird. Bei dieser Methode muss jedoch bei der Genauigkeit der Ergebnisse Abstriche gemacht werden. Da diese neue Cloaked-Area sehr groß werden kann. 
\end{itemize}
\subsubsection{anonymity-set tracing attack}
Die zweite Mögleichkeit wie ein Angreifer sein Opfer identifizieren kann ist die anonymity-set tracing attack. Hierzu muss das Opfer bei jeder Abfrage die Cloaked-Area mit anderen Partnern neu erzeugen. Hat der Angreifer nun zwei Anfragen an das LBS mitgelesen so kann er überprüfen welche Personen in der Cloaked-Area gleich geblieben sind und welche neu hinzu gekommen, bzw. die Area verlassen haben. Tritt nun der Fall ein, dass nur noch eine Person über alle Cloaked-Areas konstant bleibt, so ist diese Person das Opfer.\\
Um diesem Angriff entgegen zu wirken können drei verschiedene Verschleierungstechniken verwendet werden. Dabei ist es nicht relevant ob es sich bei den dazu verwendeten Daten um Real-Time-Daten oder Historische Daten handelt. Weiterhin wird ein sog. location anonymizer zwischen dem LBS und dem Andwender plaziert welcher die Eigentliche Position des Anwenders gegenüber dem LBS versschleiern soll.
\subsubsection{Group-Based-Approach}
Der Group-Based-Approach ist ein Ansatz welcher auf Real-Time Daten aufbaut. Hierbei stellt der Anwender seine Anfrage an den location anonymizer, welcher wiederum eine, mit k-1 sich in der Nähe befindenden Personen, Cloaked-Area erzeugt.  
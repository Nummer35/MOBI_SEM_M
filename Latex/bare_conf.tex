\documentclass[conference]{IEEEtran}
\usepackage{german}
\usepackage[utf8]{inputenc}
\usepackage[T1]{fontenc}
\hyphenation{op-tical net-works semi-conduc-tor}


\begin{document}

\title{Spatial Trajectory Privacy: Comparison of Anonymization Approaches}

\author{
\IEEEauthorblockN{Alexander Baumg"artner\\ and Tina K"ammererer\\ and Lars Taubmann}
\IEEEauthorblockA{
	Chair of Computer Science, in particular Mobile Software Systems / Mobility\\
	University of Bamberg
	Bamberg, Germany\\
	Email: firstname.lastname@uni-bamberg.de\\}
}

\maketitle


\begin{abstract}
\textbf{0,25 Seiten}
[AB]Diese Arbeit befasst sich mit verschieden geographischen Verschleierungsmethoden. Im ersten Teil werden drei Methoden Spacial Cloaking, Space Transformation und False Locations genauer betrachtet. Im zweiten Abschnitt werden die betrachteten Ansätze gegenüber gestellt und auf verschiedene Kriterien verglichen. 
\end{abstract}

Keywords—trajectory; location privacy; anonymization; spatial cloaking; dummy trajectory; location-based services

\IEEEpeerreviewmaketitle


\section{Introduction}
\textbf{0,75 - 1 Seiten}
Ziel dieses Papers ist es, verschiedene Ansätze für die Anonymisierung von Spatial Trajectories vorzustellen und zu vergleichen. Hierzu folgt zunächst ein Überblick über die Thematik Location-based Services (LBS) und die damit einhergehende Problematik des Datenschutzes, welche eine Anonymisierung der Datensätze, anschließend daran werden ergänzende und weiterführende Arbeiten vorgestellt. Im Hauptteil des Papers werden verschiedene Methoden für die Anonymisierung von Spatial Trajectories erläutert und anhand mehrerer Aspekte miteinander verglichen. \textbf{Zum Abschluss (Kurzes Stichwort für die Conclusion)}

Die Kategorie der Location-based Services umfasst eine Vielzahl an mobilen Diensten, die auf Basis der Position des Nutzers selektierte Informationen und Dienste bereit stellen. Vor allem bedingt durch die zunehmende Verbreitung von Smartphones und der damit verbundenen Möglichkeit, auf Standortdaten der Nutzer zuzugreifen, ist die LBS-Branche in Deutschland in den letzten Jahren enorm gewachsen, wie eine Studie der Bayerischen Landeszentrale für neue Medien zeigt \cite{Consulting2014}. Demnach stieg alleine die Zahl der LBS-Anbieter im Zeitraum von 2012 bis 2014 von 130 auf 927 Anbieter. Dabei werden LBS vor allem in den Bereichen Tourismus (z.B. Yelp) und Verkehr (z.B. Öffi) genutzt, ebenso auch im sozialen Bereich (z.B. Foursquare, Google Latitude), im Einkauf (z.B. Kaufda, Barcoo) und im Gebiet der pervasive Games (z.B. Ingress, Geocaching, Killer).

Allen Anwendungsgebieten zu eigen ist jedoch die Problematik des Datenschutzes, welche durch die Erfordernis und Verwendung von Nutzerdaten bedingt ist. Dies ist besonders kritisch bei kontinuierlichen LBS, welche regelmäßige Updates der Nutzerposition erfordern, da hierbei Benutzerdaten durch räumliche und zeitliche Korrelation der Trajectories erschlossen werden können \cite{Chow2011}. Dieser Aspekt ist auch für die Nutzer relevant: Laut Studie verwenden zwar 67\% der Nutzer LBS, jedoch nur 36\% davon fühlen sich dabei sicher.

Es entsteht somit ein Interessenkonflikt bei den Nutzern von LBS: Auf der einen Seite steht der Wunsch nach Privatsphäre, auf der anderen Seite der Bedarf nach möglichst akkuraten und verwertbaren Ergebnissen der Dienste, welche jedoch direkt von der Qualität der übermittelten Nutzerpositionen abhängen. Ein Kompromiss zwischen diesen beiden Aspekten lässt sich durch diverse Anonymisierungsansätze erreichen, welche die Trajectories der Nutzer verschleiern, aber dennoch einen räumlichen Kontext für die Nutzung von LBS ermöglichen. Einige dieser Ansätze werden im Folgenden genauer vorgestellt.


\section{Related Work}
\textbf{0,5 -1,5 Seiten}
Um die eigene Position vor Angreifern, oder nur vor bestimmten LBS zu verschleiern, sind in der Vergangenheit viele verschiedene Ansätze entwickelt worden. Zusätzlich zu den hier vorgestellten Ansätzen sind folgende verbreitet.
\begin{itemize}
\item{Path Confusion}
\item{Path Confusion with Mobility Prediction and Data Caching}
\item{Euler Histogram-based on Short IDs}
\end{itemize}
Ein anderer Ansatz beschäftigt sich damit, wie mit zu veröffentlichenden Daten umzugehen ist. Um hier die Möglichkeit der Anonymisierung zu bieten können folgende Ansätze näher betrachtet werden.
\begin{itemize}
\item {Clustering-based Anonymisierung}
\item{Generalization-based Anonymisierung}
\item{Suppression-based Anonymisierung}
\item{Grid-based Anonymisierung}
\end{itemize}
 

\section{Approach}
\textbf{2 Seiten pro Person}
Im Folgenden werden drei verschiedene Methoden für die Anonymisierung von Spatial Trajectories vorgestellt: Spatial Cloaking, Space Tranformation und False Locations.


\subsection{Spatial Cloaking}
Spatial Cloaking ist eine Verschleierungstaktik, bei der der Anwender seinen genauen Standort mithilfe von k-anonymity verheimlicht. Das bedeutet, dass es versucht wird eine definierte Fläche zu erzeugen. Diese Fläche soll die Eigenschaft besitzen, dass mindestens k-1 andere Personen sich in dieser aufhalten können. Um dem Anwender eine einfache Möglichkeit zu bieten, seinen Standort zu verschleiern, wird ein sog. location anonymizer zwischen den Nutzer und den Locationbased Service (LBS) geschaltet. Ein location anonymizer hat die Aufgabe den Standort des Nutzers zu verheimlichen. \\ 
Stellt der Nutzer nun eine Anfrage an einen Service wie z. B. \glqq Zeige mir alle Restaurants in der Nähe\grqq, dann schickt das Endgerät seinen genauen Standort mit Anfrage an den location anonymizer. Dieser erzeugt nun ein Gebiet, in welchem der Nutzer sich befindet und welches gleichzeitig die Sicherheitskriterien erfüllt. Bei den Sicherheitskriterien handelt es sich primär um die Größe des zu erzugenden Gebietes und wie viele weitere Personen sich in diesem aufhalten sollen. Anschließend stellt der location anonymizer die Anfrage an den vom Nutzer spezifizierten Service, jedoch wird lediglich das erzeugte Gebiet und nicht die genaue Position des Anwenders übermittelt. Als Antwort erhält der location anonymizer nun alle Restaurants in dem Gebiet und kann dem Nutzer alle daraus relevanten Informationen zur Verfügung stellen.\\ 
Wird im Folgenden von einem Gebiet oder Areal gesprochen, welches von einem location anonymizer erzeugt wurde, so spricht man von einer Cloaked-Area. Um eine Cloaked-Area zu erzeugen, gibt es verschiedene Methoden. Zum einen kann der Anonymisierer, wenn er eine sog. Cloaked-Area erzeugen möchte, bei jeder Anfrage an das LBS dieselben Partner verwenden oder zum anderen hat er die Möglichkeit sich jedes Mal neue Partner für die Erzeugung einer Cloaked-Area zu suchen. Jedoch ist es bei beiden Methoden nicht ausreichend, zu einem bestimmten Abfragezeitpunkt an den angefragten Service k-anonymität zu gewährleisten \cite{Chow2011}. Im Folgenden werden zwei verschiedene Angriffe auf das Spatial Cloaking Verfahren erläutert und wie der Nutzer sich gegen diese verteidigen kann. 
\subsubsection{Trajectory tracing Attacke} 
In \cite{Chow2011} wird ein Angriff auf Spatial Cloaking vorgestellt. Diese wird angewendet, wenn der Nutzer von Spatial Cloaking beim erzeugen einer Cloaked-Area immer die selben Partner benutzt. Der dort vorgestellte Angriff wird trajectory tracting Attacke genannt. Dabei versucht der Angreifer den Anwender anhand seiner Anfragen an das LBS zu identifizieren, indem ein Schnittpunkt zwischen dem aktuellen Gebiet und dem Gebiet, in welchem er sich seit der letzten Anfrage befinden muss, errechnet wird. 
\begin{figure}[!h]
		\centering
		\subfloat{\includegraphics[width=7cm]{Bilder/Alex/TrajectoryTracingAttacke.png}\label{figA}}
		\caption{Trajectory tracing Attacke}
		\label{fig_chow2011_traj-tracing-att}
	\end{figure}
Angenommen es wird zum Zeitpunkt t1 eine Cloaked-Area 1 mit den Nutzern A, B und C erstellt und die Bewegungsgeschwindigkeit des Opfers (im Beispiel 'Nutzer A') ist dem Angreifer bekannt. So kann der Angreifer zu jedem beliebigen nachfolgendem Zeitpunkt tn das maximale Gebiet errechnen, in welchem sich der Nutzer aufhalten kann. Dies geschieht unter der Berücksichtigung von Straßen- und Geländeverhältnissen. In Abbildung \ref{fig_chow2011_traj-tracing-att} wird diese maximale Grenze durch das rote Quadrat verdeutlicht. Daraus ergibt sich zum Zeitpunkt tn eine maximale Grenze. Zu einem Zeitpunkt t2 formt die Gruppe, bestehend aus A, B und C, eine neue Spatial-Cloaking-Area 2 und stellt erneut eine Anfrage an das LBS. Der Angreifer hat nun die Möglichkeit die maximale Grenze vom Zeitpunkt t1 mit der Spatial-Cloaking-Area 2 zu schneiden. Sollte nur ein Nutzer in dieser Schnittmenge enthalten sein, so hat der Angreifer User A erfolgreich identifiziert.\\  Der position anonymizer hat zwei Möglichkeiten sich gegen diese Art von Angriff zu wehren. Bei beiden Gegenmaßnahmen muss der Nutzer die maximale Grenze in Abhängigkeit des Zeitpunkt t1 berechnen. Anschließend kann er entweder die in \cite{Chow2011} vorgestellte \textbf{patching technique} oder die\textbf{delaying technique} anwenden. 
\begin{itemize} 
\item{delaying technique} funktioniert, indem der Anwender eine Anfrage zu einem beliebigen Zeitpunkt t2 an das LBS formuliert. Würde der Anonymisierer nun direkt die Anfrage abschicken, dann könnte der Angreifer die trajectory tracing Attacke durchführen und so den User identifizieren. Um dies zu verhindern muss die Anfrage verzögert werden. 
\begin{figure}[!h]
		\centering
		\subfloat{\includegraphics[width=3.0cm]{Bilder/Alex/DelayingTechnique.png}\label{figA}}
		\caption{Delaying technique}
		\label{fig_chow2011_delaying-tech}
\end{figure}
Die Verzögerung muss so groß gewählt werden, bis die maximale Grenze von t1 so stark angewachsen ist, dass die gebildete Cloaked-Area 2 zum Zeitpunkt t2 komplett von der maximalen Grenze überlappt wird. Wird nun die Schnittmenge aus der Cloacked-Area 2 und der maximalen Grenze zum Zeitpunkt t1 errechnet, so stellt diese das Ergebnis der maximalen Grenze zu t1 dar. Problematisch sind bei diesem Lösungsansatz jedoch, dass Anfragen an das LBS nicht zeitnah beantwortet werden können. Es muss eben gewartet werden, bis die maximale Grenze die Cloaked-Area 2 überlappt. 
\item{patching technique} kombiniert die maximale Grenze basierend auf dem Zeitpunkt t1 mit der Cloaked-Area 2, um eine neue Cloaked-Area zu erhalten. Zu sehen ist dies in Abbildung \ref{fig_chow2011_patching-tech}.
\begin{figure}[!h]
		\centering
		\subfloat{\includegraphics[width=3.0cm]{Bilder/Alex/PatchingTechnique.png}\label{figA}}
		\caption{Delaying technique}
		\label{fig_chow2011_patching-tech}
\end{figure}
Diese Cloaked-Area kann anschließend als Grundlage der Anfrage an das LBS benutzt werden. Bei dieser Methode leidet jedoch die Genauigkeit der Ergebnisse, da diese neue Cloaked-Area sehr groß werden kann, wenn sich die Personen der Cloaked Area voneinander weg bewegen.  
\end{itemize} 
\subsubsection{anonymity-set tracing Attacke} 
In \cite{chow2007} wird eine weitere Möglichkeit erläutert, spatial cloaking anzugreifen. Diese Art der Attacke wird auch anonymity-set tracing Attacke genannt. Damit sie erfolgreich durchgeführt werden kann, muss das Opfer bei jeder Anfrage an das LBS die Cloaked-Area mit anderen Partnern neu erzeugen. Hat der Angreifer nun zwei Anfragen an das LBS mitgelesen, so kann er vergleichen, welche Personen in der Cloaked-Area gleich geblieben sind und welche neu hinzugekommen, bzw. den Bereich verlassen haben. Tritt der Fall ein, dass nur noch eine einzelne Person über alle Cloaked-Areas konstant bleibt, so ist diese Person das Opfer des Sicherheitsangriffs.\\ 

Im Folgenden werden nun drei verschiedene Ansätze des Spatial Cloaking aus \cite{Chow2011} näher betrachtet. Um den eben genannten Angriffen entgegen wirken zu können, müssen diese mit den eben erläuterten Techniken verknüpft werden. Darauf wird in der folgenden Ausführung jedoch verzichtet. Bei allen Ansätzen ist es notwendig, dass ein ein location anonymizer zwischen dem LBS und dem Anwender implementiert ist. Dieser stellt die eigentliche Anfrage an das LBS, um eine Anonymität des Nutzers zu gewährleisten und um die Cloaked-Area zu erzeugen. 
\subsubsection{Group-based Ansatz} 
Der group-based Ansatz ist eine Verschleierungstaktik, welche auf Real-Time Daten aufbaut. Stellt dieselbe Person zwei Abfragen an den location anonymizer, so wird die Cloaked-Area immer mit denselben Personen erzeugt, welche auch schon bei der ersten Cloaked-Area involviert waren. So wird sichergestellt, dass die anonymity-set tracing Attacke für einen potentiellen Angreifer nicht anwendbar ist. Sucht Person A z. B. alle Restaurants in seiner Nähe, so formt der position anonymizer eine Cloaked-Area mit den Personen B und C. Damit Person A nun weitere Anfragen, wie z. B. \glqq Zeige mir alle Kaffees in der Nähe\grqq stellen kann, müssen Personen B und C ebenfalls ihre Position an den position anonymizer jederzeit zur Verfügung stellen. \\Wenn diese Art von Ansatz gewählt wird um seine Position zu verschlüsseln, so muss man jedoch Nachteile berücksichtigen. Der location anonymizer muss so die Position von allen Nutzern zu jedem Zeitpunkt kennen. Das kann vor allem bei mobilen Endgeräten Probleme mit der Akkulaufzeit verursachen.  
\subsubsection{Distortion-based Ansatz} 
Der distortion-based Ansatz ist eine Erweiterung des group-based Ansatzes. Er versucht das Problem, dass die Benutzer zwingt ihre Position immer aktuell zu halten, aus dem group-based Ansatz zu beheben. 
\begin{figure}[!h]
		\centering
		\subfloat{\includegraphics[width=7.0cm]{Bilder/Alex/DistortionBased.png}\label{figA}}
		\caption{Distortion-based Ansatz}
		\label{fig_chow2011_Distortion-Ansatz}
\end{figure}
In Abbildung \ref{fig_chow2011_Distortion-Ansatz} ist zu erkennen, wie Person A eine Anfrage an ein LBS stellt und der location anonymizer dazu eine Cloaked-Area erzeugt. Zu diesem Zeitpunkt merkt sich der location anonymizer die Geschwindigkeit jeder Person die zu dieser Gruppe gehört. Anhand der Geschwindigkeiten kann der location anonymizer nun schätzen wo sich Person B und C befinden könnten. Mithilfe dieser Schätzungen und der Position von A können nun Cloaked-Areas erzeugt werden, ohne dass Person B und C dazu gezwungen werden ihre Position zu jeder Zeit bereit zu stellen.
\subsubsection{Predication-based Ansatz} 
Dieser Ansatz benötigt keine Echtzeitinformationen von anderen Nutzern. Er ist besonders in Gebieten hilfreich, wenn sich in der Umgebung des Anfragestellers nicht genug weitere Personen befinden, um k-anonymität zu gewährleisten. Ein für ein bestimmtes Gebiet zuständiger location anoymizer speichert sich dazu die Trajectories von verschiedenen Nutzern, welche sich zu einer beliebigen Zeit eingeloggt haben. Soll der location anonymizer nun für einen Nutzer k-anonymität gewährleisten und eine Cloaked-Area erstellen, hat aber nicht genug Personen in dem entsprechendem Gebiet, so greift dieser auf zuvor gespeicherte Trajectories zurück. Dabei versucht der location anoymizer anhand der Geschwindigkeit des Nutzers vorher zu sagen, welche Footprints vergangener Aufzeichnungen sich am besten eignen um eine Cloaked-Area zu erstellen. Dies ist in Abbildung \ref{fig_chow2011_PredicationBased-Ansatz} verdeutlicht. Person A ist dabei der Nutzer des LBS und vorangegangene Trajectories wurden durch Person B und C (Blau und Pinke Punkte) erzeugt. Problematisch ist jedoch die optimale Fläche der Cloaked-Area zu berechnen, da ein Suchalgorithmus implementiert werden muss, welcher die besten Positionen von aufgezeichneten Trajectories ermittelt. Hier sollte mit einer guten Heuristic gearbeitet werden. 
\begin{figure}[!h]
		\centering
		\subfloat{\includegraphics[width=7.0cm]{Bilder/Alex/PredicationBased.png}\label{figA}}
		\caption{Predication-based Ansatz}
		\label{fig_chow2011_PredicationBased-Ansatz}
\end{figure}



\subsection{Space Transformation}
Das Bedrohungsmodel des theoretisch idealen Mix-Zone Konzeptes \cite{Beresford2003} geht davon aus, dass der Benutzer eines LBS diesem nicht traut. Um aber diesen Service dennoch in Anspruch zu nehmen,  bedient er sich eines Middleware Systems, welches als Proxy zwischen dem Nutzer und dem LBS dient. Das Middleware System unterstützt ihn dabei, seine Identität zu verstecken, indem der Nutzer nur Service Callbacks über das Middleware System und in bestimmten dafür definierten Bereichen empfängt. Dafür meldet der Nutzer Interesse an für ihn relevante LBS bei der Middleware an und diese prüft periodisch den Ort des Nutzers und bestimmt, ob für den Nutzer ein relevantes Event ausgelöst worden ist und reicht ihm die Callbacks zu angemessenen Zeitpunkten weiter. Der Middleware Proxy benutzt Pseudonyme um die Identität des Nutzers vor dem Service zu verstecken.

Da Langzeitpseudonyme es ermöglichen, die Identität des Nutzer über spezifisches Verhalten, wie das Aufhalten an bestimmten Orten wie Zuhause oder auf Arbeit, zu ermitteln, schlägt \cite{Beresford2003} einen ständigen Pseudonymwechsel vor, auch wenn der Nutzer getrackt wird. Die Middleware versorgt den Nutzer regelmäßig mit neuen, unbenutzten Pseudonymen für jede ortssensible Anwendung. Dieser Pseudonymwechsel findet innerhalb der sogenannten Mix-Zone statt. Allerdings können nur LBS genutzt werden, die Pseudonyme unterstützen.

Das Konzept der Mix-Zone wird aus dem Konzept der Mix-Nodes aus dem Bereich der anonymen Kommunikationssystemen abgeleitet \cite{Chow2011}. Die Idee, die hinter dem Mix-Node liegt ist, dass er wartet bis er k Pakete mit gleicher Länge empfangen hat und diese, bevor er sie weiter leitet, zufällig oder nach einer Metrik neu sortiert, so dass eine Unverknüpfbarkeit zwischen eingehenden und ausgehenden Paketen  entsteht.

\cite{Beresford2003} definiert eine Mix-Zone für eine Nutzergruppe als eine zusammenhängende Region mit maximaler Größe, in der keiner dieser Benutzer den Callback eines LBS erhalten hat. Für diese Nutzergruppe können mehrere unterschiedliche Mix-Zones bestehen. Das Gegenstück zu einer Mix-Zone ist die Application Zone, als Gebiet, in der ein Nutzer einen Callback empfangen hat. Der Middleware Proxy kann die Mix-Zones im Voraus oder für jede Nutzergruppe einzeln als Gebiet festlegen, in dem sich gerade kein Stück einer Application Zone befindet.

Weil standortbasierte Anwendungen in Mix-Zones keine ortsbezogenen Informationen erhalten, können sie Benutzer innerhalb der Mix-Zone nicht unterscheiden. Und so kann eine Mix-Zone die Unverknüpfbarkeit zwischen eingehenden und ausgehenden Nutzern erreichen.

\begin{figure}[!h]
		\centering
		\includegraphics[width=0.5\textwidth]{Bilder/MixZone.PNG}
		\caption{Mix-Zone Model, Quelle: \protect\cite{Palanisamy2011}}
		\label{fig_Palanisamy2011}
\end{figure}

\cite{Palanisamy2011} definiert das eine Mix-Zone Z k-anonymisierend für eine Menge von Nutzern A ist wenn folgende Bedingungen erfüllt sind:
\begin{enumerate}
\item Die Menge A hat mindestens k Elemente, $ |A| \geq k $.
\item Alle Nutzer in A müssen die Mix-Zone Z betreten haben, bevor ein Nutzer aus A die Mix-Zone Z verlässt.
\item Jeder Nutzer der Menge A die Mix-Zone durch einen Eintrittspunkt betritt $ e_{i} \in E $ und diese durch einen Austrittspunkt $ o_{i} \in O $ verlässt und dabei eine zufällige Zeitspanne in der Zone verweilt.
\item Die Übergangswahrscheinlichkeit von jeden Eintrittspunkt zu jeden Austrittspunkt folgt einer Gleichverteilung.
\end{enumerate}

\subsubsection{Mix-Zones über Straßennetze}
\cite{Palanisamy2011} beschreibt, dass die vorher betrachtete Definition der Mix-Zone davon ausgeht, dass sich die bewegenden Nutzer frei in einem euklidischen Raum bewegen, ohne dass sie von räumlichen Beschränkungen betroffen sind. Im Normalfall in der Alltagswelt der mobilen Nutzer ist dies nicht so. So sind sie zum Beispiel durch Straßennetze und Fußgängerwege in ihrer freien Bewegung eingeschränkt. Die Mix-Zones für Straßennetze werden in der Regel einer Straßenkreuzung zugeordnet. Welche Straßenkreuzung als Standort einer Mix-Zone geeignet ist, hängt gewöhnlich von einer Reihe von Faktoren ab. Diese Faktoren beinhalten die Anzahl an Straßen, die an einer Kreuzung zusammen laufen, die Geschwindigkeit der mobilen Nutzer, und Pfadbeschränkungen für mobile Nutzer, die sich innerhalb einer Mix-Zone aufhalten. Ebenso beeinflussen die ausgewählten Kreuzungen, auf denen Mix-Zones errichtet werden sollen, auch die Anzahl der Eintritts- und Austrittspunkte der Mix-Zone, da diese die eingehenden und ausgehenden Straßensegmente abbilden müssen und somit die Anzahl an Eintritts- und Austrittspunkten begrenzen. Auch liegen die Nutzer unter Beschränkungen durch das Straßennetz innerhalb einer Mix-Zone: Sie müssen sich an Geschwindigkeitsbegrenzungen halten und auch ihre freie Bewegung auf Grund von Straßen- und Gehsteigsführung ist eingeschnitten. Daraus resultiert, dass Punkt (1), das zufällig Lange Aufhalten in einer Mix-Zone, und Punkt (2), die Gleichverteilung der Übergangswahrscheinlichkeiten von Eintritts- und Austrittspunkten, in der Definition der k-Anonymität einer Mix-Zone verletzt werden. In den folgenden Absätzen wird erläutert, wie sich diese beiden Verletzungen nach \cite{Chow2011} der k-Anonymität auf den Grad der Anonymität auswirken.

\begin{figure}[!h]
		\centering
		\includegraphics[width=0.4\textwidth]{Bilder/MixZoneNetwork.PNG}
		\caption{Mix-Zone eines Straßennetwerks, Quelle: \protect\cite{Chow2011}}
		\label{fig_MixSrasse}
	\end{figure}

\paragraph{Mix-Zones ohne zufälliges Aufenthalten in ihr} Sobald Nutzer eine unbestimmte Zeit in der Mix-Zone aufhalten, stellt eine zufällige Umordnung zwischen der Eintrittsreihenfolge und der Austrittsreihenfolge eine starke Unverknüpfbarkeit zwischen alten und neuen Pseudonymen sicher. Zum Beispiel, wenn alle Nutzer eine konstante Zeit in einer Mix-Zone verbringen, würde das System dem einfachen First-In-First-Out Prinzip entsprechen. Und somit kann dem ersten verlassenden Pseudonym das entsprechende erste eintretende Pseudonym zugeordnet werde und so weiter.

\paragraph{Mix-Zones ohne gleichverteilte Übergangswahrscheinlichkeiten} Sobald die Regel gelockert wird, dass die Übergangswahrscheinlichkeit zwischen Eintritts- und Austrittspunkt gleichverteilt sind wie in der theoretischen Mix-Zone, können einige Übergange zwischen Ein- und Austrittspunkten wahrscheinlicher werden als andere. Ein Angreifer könnte dieses Wissen nutzen, um  Verknüpfungen zwischen neuen und alten Pseudonymen zu folgern. Beispielhaft könnte ein Angreifer in einen solchen Szenario Übergänge, die weniger wahrscheinlich sind, aus seiner Abbildung der Pseudonyme entsprechend dieser Übergänge eliminieren und somit das Inferieren der Pseudonymverknüpfungen verbessern.

Aus diesen beiden Schwächen leitet \cite{Palanisamy2015} die drei Arten von Attacken auf Mix-Zones über Straßennetzwerken ab: (1) Timing Attacks, (2) Transaction Attacks und (3) Combined  Timing and Transition Attacks.

1)	\textit{Timing Attack}: Bei einer Timing Attack beobachtet der Angreifer die Eintritts- $t_{in}(i)$ und die Austrittszeit $t_{out}(i)$ für jeden Nutzer der die Mix-Zone durchquert. Sobald der Angreifer erkennt das ein Nutzer i' die Mix-Zone verlässt, versucht er i' auf einen Nutzer aus dem Anonymitätsmenge Ai ab zu bilden. Der Angreifer weißt der Wahrscheinlichkeit $p_{i'\rightarrow j}$ einen Wert zu, so dass dieser der Wahrscheinlichkeit i' auf j abzubilden entspricht, wobei $j \in A$. Die Abbildungswahrscheinlichkeiten werden berechnet durch Inference basierend auf den Wahrscheinlichkeiten von den übriggebliebenen Nutzer, das diese zum Austrittszeitpunkt $t_{out}(i')$ die Mix-Zone verlassen.  Nachdem die Abbildungswahrscheinlichkeiten berechnet worden sind, kann der Angreifer die Schiefe in der Abbildungswahrscheinlichkeitsverteilung nutzen um Abbildungen mit geringer Wahrscheinlichkeit  unter Abwägung zu entfernen und so die Anzahl der Nutzer aus A, die in Betracht kommen, einzugrenzen durch die Berücksichtigung von hoch wahrscheinlichen  Abbildungen.

2)	\textit{Transition Attack}: Bei einer Transition Attack schätzt der Angreifer die Übergangswahrscheinlichkeit für jede Abbiegemöglichkeit auf Grundlage vorheriger Beobachtungen. Beim Erkennen, dass ein Nutzer i' die Mix-Zone verlässt, weißt der Angreifer der Abbildungswahrscheinlichkeit $p_{i'\rightarrow j}$  für jedes $j \in A$ auf Grundlage der Bedingten Übergangswahrscheinlichkeiten T(iseg(i), oseg(i')) und T(iseg(j), oseg(i')).  T(iseg(j), oseg(i')) ist dabei die Notation der Bedingten Wahrscheinlichkeit eines Nutzers i' der durch Straßensegment iseg(j) die Mix-Zone betreten hat unter der Bedingung diese über das Straßensegment oseg(i') verlassen hat. Die Abbildungwahrscheinlichkeiten  $p_{i'\rightarrow i}$ und $p_{i'\rightarrow j}$ unter einer Transition Attack sind daher durch \\
$p_{i'\rightarrow i} = \dfrac{T(iseg(i), oseg(i'))}{T(iseg(i), oseg(i')) + T(iseg(j), oseg(i'))} $\\
 und \\
  $p_{i'\rightarrow j} = \dfrac{T(iseg(j), oseg(i'))}{T(iseg(i), oseg(i')) + T(iseg(j), oseg(i'))} $ \\ gegeben.
  
 3)	\textit{Combined  Timing and Transition Attack}: Bei dieser Art von Angriff ist der Angreifer sowohl wissend über die Ein- und Austrittszeiten der Nutzer als auch den Übergangswahrscheinlichkeiten an Kreuzungen der gegebenen Mix-Zone des Straßennetzwerks. Der Angreifer kann  daher die Abbildungswahrscheinlichkeit $p_{i'\rightarrow j}$ für jedes $j \in A$ basierend auf den Wahrscheinlichkeiten des Nutzers j zum Zeitpunkt $t_{out}(i')$ die Mix-Zone verlässt und den Bedingten Übergangswahrscheinlichkeiten $T(iseg(j), oseg(i'))$ schätzen. 
 
 \begin{figure}[!h]
 	\centering
 	\includegraphics[width=0.4\textwidth]{Bilder/nonrectangularMix.PNG}
 	\caption{Non-Rectangular Mix-Zone, Quelle: \protect\cite{Chow2011}}
 	\label{fig_MixSrasseNon}
 \end{figure}
 
 Da eine einfache rechteckige Mix-Zone um eine Straßenkreuzung diesen Attacken nicht ausreichend stand hält, haben [Palanisamy2015, Palanisamy2011] die Time Window Bounded Non-Rectangular Mix-Zones entworfen. Dieser Typ von Mix-Zone besteht aus mehreren rechteckigen Teilstücken.  Diese beginnen von der Mitte der Kreuzung und liegen nur auf den ausgehenden Straßensegmenten der Kreuzung  (siehe Abbildung \ref{fig_MixSrasseNon}). Dieser Ansatz wird im Folgenden als Non-Rectangular Ansatz bezeichnet. Dieser Ansatz schützt besser vor Timing Attacks, die auf der Heterogenität der  Geschwindigkeitsverteilung auf den Straßensegmenten basieren. Es wird angenommen, das sie Anonymitätsmenge für jeden Nutzer i alle Nutzer beinhaltet, die die Mix-Zone in einem Zeitfenster mit dem Intervall $| t_{in}(i)-\tau_{1}|$ bis $|t_{in}(i)+\tau_{2}|$ betreten haben. $t_{in}(i)$ ist der Eintrittszeitpunkt des Nutzers i und ta1 und ta2 werden als so kleine Werte gewählt, dass das Zeitfenster sicherstellt, dass die Anonymitätsmenge von i nur Nutzer ähnlich naher Ankunftszeit beinhaltet. Die Länge der Mix-zone auf den ausgehenden Straßensegmenten  wird basierend auf der Geschwindigkeit des Straßensegments, der Größe des Zeitfensters und dem benötigten Anonymitätsgrad bestimmt.     


\subsection{False Locations}
Die Verwendung von Dummy Trajectories hat jedoch auch einige Nachteile. Zum einen ensteht ein Overhead bei jeder Anfrage, da sowohl beim Nutzer und bei eventueller Middleware, als auch bei den LBS, viele zusätzliche Ressourcen für die Erstellung und Bearbeitung der Dummies benötigt werden.

Dummy users might have to control physical objects—opening
and closing doors, for example—or purchase services with
electronic cash. Furthermore, realistic dummy user movements
are much more difficult to construct than dummy messages.


\subsubsection{Pseudonyme und Middleware}
Ein anderer Ansatz, um die Privatsphäre von LBS-Nutzern zu gewährleisten, wird durch die Kombination von Dummy Locations und einem Application Server realisiert \cite{Sahu2012}. Hierbei gibt es einige Ähnlichkeiten mit den Mix Zones \cite{Beresford2003}, bei denen Nutzer über Middleware anonymisiert werden, indem sie für die Aufenthaltsdauer in einem definierten räumlichen Bereich ein Pseudonym annehmen und auf Positionsupdates verzichten. Somit kann der Nutzer nicht von anderen Personen in der Mix Zone differenziert werden und es ist ebenso keine Verbindung zwischen dem Eintritt und Austritt eines Nutzers aus der Mix Zone herstellbar. Es ergibt sich jedoch der Nachteil, dass durch die nicht akkurate Positionsangabe auch die Qualität der LBS abnimmt. 
Bei dem vorgestellten Ansatz wird ebenfalls Middleware eingesetzt, die dem Nutzer Pseudonyme zuweist, die nach einem bestimmten Zeitraum geändert werden. Anders als bei den Mix Zones übermittelt der Nutzer jedoch weiterhin Positionsupdates an die LBS, um einen bestmöglichen Service zu erhalten. Jedoch ist dadurch der alleinige Einsatz von mehreren aufeinander folgenden Pseudonymen für die Privatsphäre nicht ausreichend, da durch Inferenz eine Verknüpfung zwischen diesen hergestellt werden könnte. Deshalb generiert der Nutzer eine Reihe von Dummy Locations, welche zusätzlich zu der realen Position über die Middleware an die LBS übermittelt werden. Damit steht den LBS eine genaue Position zur Verfügung, um Ergebnisse für die Anfrage zu generieren, gleichzeitig bleibt die Identität des Nutzers jedoch unbekannt. Aus den Ergebnissen der Anfrage kann der Nutzer nun die für ihn relevanten Informationen selektieren.
Abhängig von der Anzahl der generierten Dummies K und den mit der Zeit wechselnden Pseudonyme N für jeden Nutzer in einer Zeit T entstehen somit K$^{N}$ verschiedene Pfade für jeden Benutzer. Die Variable K kann somit Abhängig von dem gewünschten Anonymitätsgrad und der zur Verfügung stehenden Processing Power gewählt werden.



\section{Analysis}
\textbf{1 - 3 Seiten}
\begin{table}[!t]
\renewcommand{\arraystretch}{1.3}
\caption{Vergleich verschiedener Anonymisierungsans"atze Part 1}
\label{table:vergleich1}
\centering
    \begin{tabular}{l|lll}
    	~                    & LBS I / LBS II & Processing Power & Overhead \\ \hline
    	Spatial Cloaking     & ~              & ~                & ~        \\
    	Space Transformation & ~              & ~                & ~        \\
    	False Locations      & ~              & ~                & ~        \\
    \end{tabular}
\end{table}

\begin{table}[!t]
\renewcommand{\arraystretch}{1.3}

\caption{Vergleich verschiedener Anonymisierungsans"atze Part 2}
\label{table:vergleich2}
	\centering
    \begin{tabular}{l|lll}
    	~                    & Relocatable & Involved Parties & Trusted Services \\ \hline
    	Spatial Cloaking     & ~           & ~                & ~                \\
    	Space Transformation & ~           & ~                & ~                \\
    	False Locations      & ~           & ~                & ~                \\
    \end{tabular}
\end{table}


\section{Conclusion}
\textbf{0,5 - 1 Seiten}
Dieses Paper stellt die drei Ansätze (1) Spatial Cloaking, (2) Mix-Zone und False Locations zum Schutz der Spatial Trajectory Privacy vor und vergleicht diese anschließend.

\begin{enumerate}
	\item In der Sektion des Spatial Cloaking werden die für diesen Ansatz spezielle Angriffsarten auf die Trajectory Privacy von LBS-Nutzern Trajectory Tracing Attacke und die Anonymity-Set Attacke auf gezeigt und die daraus resultierenden Lösungsansätze  Group-based, Distortion-based und Predication-based erläutert.
	\item Im Abschnitt der Mix-Zone wird das Konzept der Mix-Zone präsentiert und erläutert, unter welchen Bedingungen eine Mix-Zone die k-Anonymity hält. Desweiteren werden auf Mix-Zones über Straßennetzen eingegangen, weil diese auf Grund der Einschränkung ihrer Benutzer in Bezug auf Bewegungsfreiheit und Geschwindigkeit spezielle Eigenschaften besitzen müssen.
	\item Beim Ansatz der False Locations werden zu Beginn die für diesen Ansatz geeigneten Privatsphären-Parameter Short-term, Long-term und Distance Deviation vorgestellt. Anschließend werden die Ansätze Realitätsnahe Dummy Trajectories, Front-end Modul und Pseudonyme und Middleware diskutiert. Abschließend werden die Dummy-Generation-Schemata Random Pattern Scheme, Intersection Patern-based,  Moving in a Neighborhood, Moving in a Limited Neighborhood, Circle-Based und Grid-Based aufgezeigt.
\end{enumerate}
Abschließend werden die verschieden Ansätze zum Schutz der Spatial Trajectory Privacy an Hand der Kriterien unterstützte LBS-Kategorien (I \& II) , benötigter Rechenleistung, Nachrichteninformations-Overhead, Relocation, Involved Parties und benötigter Trusted Services verglichen. Dabei stellte sich heraus, dass nur bei Ansätzen des Types False Locations nur der Benutzer selbst benötigt wurde, um dessen Anonymität zu gewährleisten, im Gegensatz brauchten den die Ansätze Mix-Zone und Spatial Cloaking eine bestimmte Anzahl an Usern, um deren Privacy zu schützen, weil sie auf dem Prinzip der k-Anonymity beruhen. Desweiteren ist auch auffällig, das die Ansätze False Location I und False Location II (siehe Tabelle \ref{table:vergleich1}) keinen Trusted Service benötigen um ihren Dienst zu erfüllen. Auf Grund der Benutzung von Pseudonymen können Mix-Zone und der False Location Ansatz III nur LBS’es der Kategorie II im Gegensatz zu den Anderen Ansaätzen, die Kategorie I und II unterstützen, bedienen. Auffällig ist auch noch das Mix-Zones nicht einfach relokasiert werden können, da sie für ihre bestimmte räumliche Region angepasst sind. Auch ist bei ihnen der Nachrichteninformations-Overhead geringer als bei den anderen Ansätzen, weil der Trusted Service zwischen LBS und User nur die Anfragen und Antworten weiterleiten muss, im Gegensatz zu Spatial Cloaking, bei diesem Ansatz ganzer Koordinatenregion übertragen werden muss oder den False Location Ansätzen bei denen Dummy-Anfragen und aus denen resultierenden Antworten generiert werden.    
The conclusion goes here.



\bibliographystyle{IEEEtran}
\bibliography{Literature}



\end{document}
Dieses Paper stellt die drei Ansätze (1) Spatial Cloaking, (2) Mix-Zone und False Locations zum Schutz der Spatial Trajectory Privacy vor und vergleicht diese anschließend.

\begin{enumerate}
	\item In der Sektion des Spatial Cloaking werden die für diesen Ansatz spezielle Angriffsarten auf die Trajectory Privacy von LBS-Nutzern Trajectory Tracing Attacke und die Anonymity-Set Attacke auf gezeigt und die daraus resultierenden Lösungsansätze  Group-based, Distortion-based und Predication-based erläutert.
	\item Im Abschnitt der Mix-Zone wird das Konzept der Mix-Zone präsentiert und erläutert, unter welchen Bedingungen eine Mix-Zone die k-Anonymity hält. Desweiteren werden auf Mix-Zones über Straßennetzen eingegangen, weil diese auf Grund der Einschränkung ihrer Benutzer in Bezug auf Bewegungsfreiheit und Geschwindigkeit spezielle Eigenschaften besitzen müssen.
	\item Beim Ansatz der False Locations werden zu Beginn die für diesen Ansatz geeigneten Privatsphären-Parameter Short-term, Long-term und Distance Deviation vorgestellt. Anschließend werden die Ansätze Realitätsnahe Dummy Trajectories, Front-end Modul und Pseudonyme und Middleware diskutiert. Abschließend werden die Dummy-Generation-Schemata Random Pattern Scheme, Intersection Patern-based,  Moving in a Neighborhood, Moving in a Limited Neighborhood, Circle-Based und Grid-Based aufgezeigt.
\end{enumerate}
Abschließend werden die verschieden Ansätze zum Schutz der Spatial Trajectory Privacy an Hand der Kriterien unterstützte LBS-Kategorien (I \& II) , benötigter Rechenleistung, Nachrichteninformations-Overhead, Relocation, Involved Parties und benötigter Trusted Services verglichen. Dabei stellte sich heraus, dass nur bei Ansätzen des Types False Locations nur der Benutzer selbst benötigt wurde, um dessen Anonymität zu gewährleisten, im Gegensatz brauchten den die Ansätze Mix-Zone und Spatial Cloaking eine bestimmte Anzahl an Usern, um deren Privacy zu schützen, weil sie auf dem Prinzip der k-Anonymity beruhen. Desweiteren ist auch auffällig, das die Ansätze False Location I und False Location II (siehe Tabelle \ref{table:vergleich1}) keinen Trusted Service benötigen um ihren Dienst zu erfüllen. Auf Grund der Benutzung von Pseudonymen können Mix-Zone und der False Location Ansatz III nur LBS’es der Kategorie II im Gegensatz zu den Anderen Ansaätzen, die Kategorie I und II unterstützen, bedienen. Auffällig ist auch noch das Mix-Zones nicht einfach relokasiert werden können, da sie für ihre bestimmte räumliche Region angepasst sind. Auch ist bei ihnen der Nachrichteninformations-Overhead geringer als bei den anderen Ansätzen, weil der Trusted Service zwischen LBS und User nur die Anfragen und Antworten weiterleiten muss, im Gegensatz zu Spatial Cloaking, bei diesem Ansatz ganzer Koordinatenregion übertragen werden muss oder den False Location Ansätzen bei denen Dummy-Anfragen und aus denen resultierenden Antworten generiert werden.    
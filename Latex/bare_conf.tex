\documentclass[conference]{IEEEtran}
\usepackage{german}
\usepackage[utf8]{inputenc}
\usepackage[T1]{fontenc}
\hyphenation{op-tical net-works semi-conduc-tor}


\begin{document}

\title{Spatial Trajectory Privacy: Comparison of Anonymization Approaches}

\author{
\IEEEauthorblockN{Alexander Baumg"artner\\ and Tina K"ammererer\\ and Lars Taubmann}
\IEEEauthorblockA{University of Bamberg\\
96047 Bamberg, Germany\\
Email: firstname.lastname@uni-bamberg.de\\}
}

\maketitle


\begin{abstract}
\textbf{0,25 Seiten}
[AB]Diese Arbeit befasst sich mit verschieden geographischen Verschleierungsmethoden. Im ersten Teil werden drei Methoden Spacial Cloaking, Space Transformation und False Locations genauer betrachtet. Im zweiten Abschnitt werden die betrachteten Ansätze gegenüber gestellt und auf verschiedene Kriterien verglichen. 
\end{abstract}

\IEEEpeerreviewmaketitle


\section{Introduction}
\textbf{0,75 - 1 Seiten}
Ziel dieses Papers ist es, verschiedene Ansätze für die Anonymisierung von Spatial Trajectories vorzustellen und zu vergleichen. Hierzu folgt zunächst ein Überblick über die Thematik Location-based Services (LBS) und die damit einhergehende Problematik des Datenschutzes, anschließend daran werden ergänzende und weiterführende Arbeiten vorgestellt. Im Hauptteil des Papers werden verschiedene Methoden für die Anonymisierung von Spatial Trajectories erläutert und anhand mehrerer Aspekte miteinander verglichen. Zum Abschluss (Kurzes Stichwort für die Conclusion)

Die Kategorie der Location-based Services umfasst eine Vielzahl an mobilen Diensten, die auf Basis der Position des Nutzers selektierte Informationen und Dienste bereit stellen. Vor allem bedingt durch die zunehmende Verbreitung von Smartphones und der damit verbundenen Möglichkeit, auf Standortdaten der Nutzer zuzugreifen, ist die LBS-Branche in Deutschland in den letzten Jahren enorm gewachsen, wie eine Studie der Bayerischen Landeszentrale für neue Medien zeigt \cite{Consulting2014}. Demnach stieg alleine die Zahl der LBS-Anbieter im Zeitraum von 2012 bis 2014 von 130 auf 927 Anbieter. Dabei werden LBS vor allem in den Bereichen Tourismus (z.B. Yelp) und Verkehr (z.B. Öffi) genutzt, aber auch im sozialen Bereich (z.B. Foursquare, Google Latitude), im Einkauf (z.B. Kaufda, Barcoo) und im Gebiet der pervasive Games (z.B. Ingress, Geocaching, Killer).

Allen Anwendungsgebieten zu eigen ist jedoch die Problematik des Datenschutzes, welche durch die Erfordernis und Verwendung von Nutzerdaten. Dies ist besonders kritisch bei kontinuierlichen LBS, welche regelmäßige Updates der Nutzerposition erfordern, da hierbei Benutzerdaten durch räumliche und zeitliche Korrelation der Trajectories erschlossen werden können \cite{Chow2011}. Dieser Aspekt ist auch für die Nutzer relevant: Laut Studie verwenden zwar 67\% der Nutzer LBS, aber nur 36\% davon fühlen sich dabei sicher.





\hfill mds
 
\hfill January 11, 2007

\section{Related Work}
\textbf{0,5 -1,5 Seiten}
RelatedWork
 

\section{Approach}
\textbf{2 Seiten pro Person}
Im Folgenden werden drei verschiedene Methoden für die Anonymisierung von Spatial Trajectories vorgestellt: Spatial Cloaking, Space Tranformation und False Locations.


\subsection{Spatial Cloaking}
Spatial Cloaking ist eine Verschleierungstaktik, bei der der Anwender seinen genauen Standort mithilfe von k-anonymity [1] verheimlicht. Das bedeutet, dass versucht wird eine definierte Fläche zu erzeugen. Diese Fläche soll die Eigenschaft besitzen, dass mindestens k-1 andere Personen sich in dieser aufhalten können. Um dem Anwender eine einfache Möglichkeit zu bieten, seinen Standort zu verschleiern, wird ein sog. locatin anonymizer zwischen den Nutzer und den Locationbased Service (LBS) geschaltet. \\ Stellt der Nutzer nun eine Anfrage an einen Service wie z. B. \"Zeige mir alle Restaurants in der Nähe\", dann schickt das Endgerät seinen genauen Standort mit Anfrage an den Positionsanonymisierer. Dieser erzeugt nun ein Areal, in welchem der Nutzer sich befindet und welches gleichzeitig die Sicherheitskriterien erfüllt. Anschließend stellt der location anonymizer die Anfrage an den vom Nutzer spezifizierten Service, jedoch wird lediglich das Areal und nicht die genaue Position des Anwenders übermittelt. Als Antwort erhält der Positionsanonymisierer nun alle Restaurants in dem Gebiet und kann dem Nutzer alle daraus relevanten Informationen zur Verfügung stellen.\\ Wird im Folgenden von einem Gebiet oder Areal gesprochen, welches von einem Positonsanonyisierer erzeugt wurde, wird[so spricht man von einem Cloaked...] dieses Cloaked-Area genannt. Um eine Cloaked-Area zu erzeugen gibt es verschiedene Methoden. Zum einen [wo ist das zum anderen?] kann der Anonymisierer, wenn er eine sog. Cloaked-Area erzeugen möchte, bei jeder Anfrage an das LBS dieselben Partner verwenden oder [hier zum anderen?] er hat die Möglichkeit sich jedes Mal neue Partner für die Erzeugung einer Cloaked-Area zu suchen. Jedoch ist es bei beiden Methoden nicht ausreichend, zu einem bestimmten Abfragezeitpunkt an den angefragten Service k-anonymität zu gewährleisten [Zitat]. Im Folgenden werden zwei verschiedene Angriffe (trajectory tracing attack und anonymity-set tracing attack) auf das Spatial Cloaking Verfahren erläutert und wie der Nutzer sich gegen diese verteidigen kann. 
\subsubsection{Trajectory tracing Attacke} 
Die trajectory tracing Attacke kann verwendet werden, sollte der Anonymisierer immer dieselben Partner zur Erzeugung einer Cloaked-Area verwenden.[Es ist sinnvoll die trajectory tracting Attacke zu verwenden, wenn der Anonymisierer immer dieselben Partner zur Erzeugung einer Cloaked-Area verwendet] Dabei versucht der Angreifer das Opfer anhand seiner Anfragen an das LBS zu identifizieren, indem ein Schnittpunkt zwischen dem aktuellen Gebiet und dem Gebiet, in welchem er sich seit der letzten Anfrage befinden muss, errechnet wird. Angenommen es wird zum Zeitpunkt t1 eine Cloaked-Area a1 mit den Usern A, B und C erstellt und die Bewegungsgeschwindigkeit von Opfer A ist dem Angreifer bekannt. So kann der Angreifer zu jedem beliebigen nachfolgendem Zeitpunkt tn das maximale Gebiet errechnen, in welchem sich der Nutzer aufhalten kann. Dies muss unter der Berücksichtigung von Straßen- und Geländeverhältnissen geschehen. Daraus ergibt sich zum Zeitpunkt tn eine maximale Grenze. Zu einem Zeitpunkt t2 formt die Gruppe eine neue Spatial-Cloaking-Area a2 und stellt erneut Anfragen an das LBS. Der Angreifer hat nun die Möglichkeit die maximale Grenze vom Zeitpunkt t1 mit a2 zu schneiden. Sollte nur ein Nutzer in dieser Schnittmenge enthalten sein, so hat der Angreifer User A erfolgreich identifiziert.\\  Der Anonymisierer hat zwei Möglichkeiten sich gegen diese Art von Angriff zu wehren. Bei beiden Gegenmaßnahmen muss der Nutzer die maximale Grenze in Abhängigkeit des Zeitpunkt t1 berechnen. Anschließend kann er entweder die \textbf{patching technique} oder die\textbf{delaying technique} anwenden. 
\begin{itemize} 
\item{delaying technique} funktioniert, indem der Anwender eine Anfrage zu einem beliebigen Zeitpunkt t2 an die LBS formuliert. Würde der Anonymisierer nun direkt die Anfrage abschicken, dann könnte der Angreifer die trajectory tracing Attacke durchführen und so den User identifizieren. [Um dies zu verhindern?]Deswegen muss die Anfrage verzögert werden. Die Zeit der Verzögerung ist so groß[Die Verzögerung muss so groß gewählt werden, damit die maximale Grenze von t1 so stark angewachsen ist, bis...], bis die maximale Grenze, bezogen auf t1, so stark angewachsen ist bis diese die gebildete Cloaked-Area a2 zum Zeitpunkt t2 komplett überlappt. Wird nun die Schnittmenge aus a2 und der maximalen Grenze zu t1 errechnet,[so stellt diese das Ergebnis der maximalen Grenze zu t1 dar] ist das Ergebnis die maximale Grenze zu t1. Problematisch sind bei diesem Lösungsansatz jedoch, dass Anfragen an das LBS nicht zeitnah beantwortet werden können. Es gibt immer eine gewisse Wartezeit. \item{patching technique} kombiniert die maximale Grenze basierend auf t1 mit der Cloaked-Area a2, um eine neue Cloaked-Area zu erhalten, welche als Grundlage bei der Anfrage des LBS benutzt wird. Bei dieser Methode leidet jedoch die Genauigkeit der Ergebnisse, da diese neue Cloaked-Area sehr groß werden kann, wenn sich die Personen der Cloaked Area voneinander weg bewegen.  
\end{itemize} 
\subsubsection{anonymity-set tracing Attacke} 
Die zweite Möglichkeit wie ein Angreifer sein Opfer identifizieren kann, ist mithilfe der anonymity-set tracing Attacke. Hierzu muss das Opfer bei jeder Anfrage an das LBS die Cloaked-Area mit anderen Partnern neu erzeugen. Hat der Angreifer nun zwei Anfragen an das LBS mitgelesen, so kann er vergleichen, welche Personen in der Cloaked-Area gleich geblieben sind und welche neu hinzugekommen, bzw. den Bereich verlassen haben. Tritt der Fall ein, dass nur noch eine einzelne Person über alle Cloaked-Areas konstant bleibt, so ist diese Person das Opfer[der Sicherheitsattacke?].\\ Um diesem Angriff entgegen zu wirken, können drei verschiedene Verschleierungstaktiken verwendet werden. Dabei ist es nicht relevant, ob es sich bei den dazu verendeten Daten um Real-Time-Daten handelt, also ob sich zum Zeitpunkt der Erstellung einer Cloaked-Area weitere Personen in der Nähe befinden oder nicht oder[ob es sich um] historische Daten handelt. Außerdem muss ein sog. location anonymizer zwischen dem LBS und dem Anwender platziert werden. Dieser stellt die eigentliche Anfrage an das LBS, um eine Anonymität des Nutzers zu gewährleisten. 
\subsubsection{Group-based Ansatz} 
Der group-based Ansatz ist eine Verschleierungstaktik, welcher[welche?] auf Real-Time Daten aufbaut. Stellt dieselbe Person eine zwei [eine oder zwei?]verschiedene Abfragen an den location anonymizer, so wird die Cloaked-Area immer mit denselben Personen erzeugt, welche auch schon bei der ersten Cloaked-Area involviert waren. So wird sichergestellt, dass die anonymity-set tracing Attacke für einen potentiellen Angreifer nicht anwendbar ist.\\Wenn diese Art von Ansatz gewählt wird um seine Position zu verschlüsseln, so muss man jedoch Nachteile berücksichtigen. Der location anonymizer muss so die Position von allen Nutzern zu jedem Zeitpunkt kennen. Das kann vor allem bei mobilen Endgeräten Probleme mit der Akkulaufzeit verursachen.  
\subsubsection{Distortion-based Ansatz} 
Der distortion-based Ansatz ist eine Erweiterung des group-based Ansatzes. Er versucht das Problem des ständigen online seins aus dem group-based Ansatz zu beheben. Hierzu versucht der Anonymisierer die Position von Nutzern vorher zu sagen. Dazu benutzt dieser die letzte bekannte Position und Geschwindigkeit der einzelnen Personen. Nun muss nur der Nutzer, welcher wirklich eine Anfrage stellt, seinen Standort dem Anonymisierer verraten und es kann dadurch trotzdem eine k-anonymität gewährleistet werden. [Umgangssprache, satz besser formulieren] 
\subsubsection{Predication-based Ansatz} 
Dieser Ansatz benötigt keine Echtzeitinformationen von anderen Nutzern. Ein für ein bestimmtes Gebiet zuständiger Anonymisierer speichert sich dazu die Standorte von verschiedenen Nutzern, welche sich zu einer beliebigen Zeit eingeloggt haben. Soll der location anonymizer nun für einen Nutzer k-anonymität gewährleisten, hat aber nicht genug Personen in dem entsprechendem Gebiet, so greift dieser auf zuvor gespeicherte trajectorys zurück. Problematisch ist jedoch die optimale Fläche der Cloaked-Area zu berechnen, da ein Suchalgorithmus implementiert werden muss, welcher die besten Positionen von aufgezeichneten trajectorys ermittelt. Hier sollte mit einer guten Heuristic gearbeitet werden. 



\subsection{Space Transformation}
Lars


\subsection{False Locations}
Die Verwendung von Dummy Trajectories hat jedoch auch einige Nachteile. Zum einen entsteht ein Overhead bei jeder Anfrage, da sowohl beim Nutzer und bei eventueller Middleware, als auch bei den LBS, viele zusätzliche Ressourcen für die Erstellung und Bearbeitung der Dummies benötigt werden. Im Gegensatz zu Dummy Positions ist die Generierung von realistischen Dummy Trajectories, die nicht von realen Trajectories unterschieden werden können, relativ anspruchsvoll \cite{Beresford2003}.


\subsubsection{Pseudonyme und Middleware} \label{subsubsection:pseudomiddle}
Ein anderer Ansatz, um die Privatsphäre von LBS-Nutzern zu gewährleisten, wird durch die Kombination von Dummy Locations und einem Application Server realisiert \cite{Sahu2012}. Hierbei gibt es einige Ähnlichkeiten mit den Mix Zones \cite{Beresford2003}, bei denen Nutzer über Middleware anonymisiert werden, indem sie für die Aufenthaltsdauer in einem definierten räumlichen Bereich ein Pseudonym annehmen und auf Positionsupdates verzichten. Somit kann der Nutzer nicht von anderen Personen in der Mix Zone differenziert werden und es ist ebenso keine Verbindung zwischen dem Eintritt und Austritt eines Nutzers aus der Mix Zone herstellbar. Es ergibt sich jedoch der Nachteil, dass durch die nicht akkurate Positionsangabe auch die Qualität der LBS abnimmt. 
Bei dem vorgestellten Ansatz wird ebenfalls Middleware eingesetzt, die dem Nutzer Pseudonyme zuweist, die nach einem bestimmten Zeitraum geändert werden. Anders als bei den Mix Zones übermittelt der Nutzer jedoch weiterhin Positionsupdates an die LBS, um einen bestmöglichen Service zu erhalten. Jedoch ist dadurch der alleinige Einsatz von mehreren aufeinander folgenden Pseudonymen für die Privatsphäre nicht ausreichend, da durch Inferenz eine Verknüpfung zwischen diesen hergestellt werden könnte. Deshalb generiert der Nutzer eine Reihe von Dummy Locations, welche zusätzlich zu der realen Position über die Middleware an die LBS übermittelt werden. Damit steht den LBS eine genaue Position zur Verfügung, um Ergebnisse für die Anfrage zu generieren, gleichzeitig bleibt die Identität des Nutzers jedoch unbekannt. Aus den Ergebnissen der Anfrage kann der Nutzer nun die für ihn relevanten Informationen selektieren.
Abhängig von der Anzahl der generierten Dummies K und den mit der Zeit wechselnden Pseudonyme N für jeden Nutzer in einer Zeit T entstehen somit K$^{N}$ verschiedene Pfade für jeden Benutzer. Die Variable K kann somit Abhängig von dem gewünschten Anonymitätsgrad und der zur Verfügung stehenden Processing Power gewählt werden.



\section{Analysis}
\textbf{1 - 3 Seiten}
\begin{table*}[!ht]
\renewcommand{\arraystretch}{1.3}
\caption{Vergleich verschiedener Anonymisierungsansätze}
\label{table:vergleich1}
\centering
    \begin{tabular}{{l|llllll}}
    	~                    								& LBS I / LBS II   & Processing Power & Overhead 	& Relocatable 	& Involved Parties & Trusted Services \\ \hline
    	Spatial Cloaking     								& ~                & ~                & ~        	& ~           	& ~                & ~                \\
    	Space Transformation 								& ~                & ~                & ~        	& ~           	& ~                & ~                \\
	    False Locations \ref{subsubsection:realdummy}     	& Ja / Ja          & Middle   		  & Middle      & Ja           	& User*            & Nein			  \\
    	False Locations \ref{subsubsection:pseudomiddle}    & Nein / Ja        & High     		  & High        & Ja            & User             & Middleware	      \\
    \end{tabular}
\end{table*}


\section{Conclusion}
\textbf{0,5 - 1 Seiten}
Conclusion
The conclusion goes here.



\bibliographystyle{IEEEtran}
\bibliography{Literature}



\end{document}